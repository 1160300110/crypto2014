% presentation
\documentclass{beamer}
\usetheme[height=7mm]{Rochester}
\usecolortheme{rose}

% handout

%\documentclass[handout]{beamer}
%\usepackage{pgfpages} \pgfpagesuselayout{8 on 1}[a4paper]

%\documentclass[mathserif]{article}
%\usepackage{beamerarticle}

\usepackage{amsmath}
\usepackage{comment}
\usepackage{amssymb,amsfonts}
\usepackage[T1]{fontenc}
\usepackage{lmodern}
\usepackage{tikz}
\usepackage{simpsons}
\usepackage{marvosym}
\usepackage{color}
\usepackage{multirow}
\usepackage{pgffor}
\usepackage[slide,algoruled,titlenumbered,vlined,noend,linesnumbered,]{algorithm2e}

\usefonttheme{structurebold}

\setbeamertemplate{footline}[frame number]
\setbeamertemplate{navigation symbols}{}
\setbeamerfont{smallverb}{size*={73}}
\usefonttheme[onlymath]{serif}
\setbeamertemplate{theorems}[numbered]
\newtheorem{construction}[theorem]{Construction}
\newtheorem{proposition}[theorem]{Proposition}

\AtBeginSection[] { 
  \begin{frame} 
    \frametitle{Content} 
    \tableofcontents[currentsection]
  \end{frame} 
  \addtocounter{framenumber}{-1} 
}

\usetikzlibrary[shapes.arrows]
\usetikzlibrary{shapes.geometric}
\usetikzlibrary{backgrounds}
\usetikzlibrary{positioning}
\usetikzlibrary{calc}
\usetikzlibrary{intersections}
\usetikzlibrary{fadings}
\usetikzlibrary{decorations.footprints}
\usetikzlibrary{patterns}
\usetikzlibrary{shapes.callouts}
\usetikzlibrary{fit}
%handout

\providecommand{\abs}[1]{\lvert#1\rvert}

\tikzset{every picture/.style={line width=1pt,show background rectangle},background rectangle/.style={fill=blue!10,rounded corners=2ex}}

\author{Yu Zhang}
\institute{Harbin Institute of Technology}
\date[Crypt'16A]{Cryptography, Autumn, 2016}

%% presentation
\documentclass{beamer}
\usetheme[height=7mm]{Rochester}
\usecolortheme{rose}

% handout

%\documentclass[handout]{beamer}
%\usepackage{pgfpages} \pgfpagesuselayout{8 on 1}[a4paper]

%\documentclass[mathserif]{article}
%\usepackage{beamerarticle}

\usepackage{amsmath}
\usepackage{comment}
\usepackage{amssymb,amsfonts}
\usepackage[T1]{fontenc}
\usepackage{lmodern}
\usepackage{tikz}
\usepackage{simpsons}
\usepackage{marvosym}
\usepackage{color}
\usepackage{multirow}
\usepackage{pgffor}
\usepackage[slide,algoruled,titlenumbered,vlined,noend,linesnumbered,]{algorithm2e}

\usefonttheme{structurebold}

\setbeamertemplate{footline}[frame number]
\setbeamertemplate{navigation symbols}{}
\setbeamerfont{smallverb}{size*={73}}
\usefonttheme[onlymath]{serif}
\setbeamertemplate{theorems}[numbered]
\newtheorem{construction}[theorem]{Construction}
\newtheorem{proposition}[theorem]{Proposition}

\AtBeginSection[] { 
  \begin{frame} 
    \frametitle{Content} 
    \tableofcontents[currentsection]
  \end{frame} 
  \addtocounter{framenumber}{-1} 
}

\usetikzlibrary[shapes.arrows]
\usetikzlibrary{shapes.geometric}
\usetikzlibrary{backgrounds}
\usetikzlibrary{positioning}
\usetikzlibrary{calc}
\usetikzlibrary{intersections}
\usetikzlibrary{fadings}
\usetikzlibrary{decorations.footprints}
\usetikzlibrary{patterns}
\usetikzlibrary{shapes.callouts}
\usetikzlibrary{fit}
%handout

\providecommand{\abs}[1]{\lvert#1\rvert}

\tikzset{every picture/.style={line width=1pt,show background rectangle},background rectangle/.style={fill=blue!10,rounded corners=2ex}}

\author{Yu Zhang}
\institute{Harbin Institute of Technology}
\date[Crypt'16A]{Cryptography, Autumn, 2016}

%\input{1introduction.tex}
%\input{2perfectlysecret.tex}
%\input{3privatekey.tex}


\title{Introduction}

\begin{document}
\maketitle
\begin{frame}
\frametitle{Outline}
\tableofcontents
\end{frame}
\section{Cryptography and Modern Cryptography}
\begin{frame}\frametitle{What is Cryptography?}
\begin{itemize}
\item \textbf{Cryptography}: from Greek \emph{krypt\'os}, ``hidden, secret''; and \emph{gr\'{a}phin}, ``writing''.
\item \textbf{Cryptography}: the art of writing or solving codes.\\ (Concise oxford dictionary 2006)
\item \textbf{Codes}: a system of prearranged signals, especially used to ensure secrecy in transmitting messages. \\ (\emph{code word} in cryptography)
\item \textbf{1980s}: from Classic to Modern; from Military to Everyone.
\item \textbf{Modern cryptography}: the scientific study of techniques for securing digital information, transactions, and distributed computations.
\end{itemize}
\end{frame}
\section{The Setting of Private-Key Encryption}
\begin{frame}\frametitle{Private-Key Encryption}
\begin{itemize}
\item \textbf{Goal}: to construct a \textbf{ciphers} (encryption schemes) for providing secret communication between two parties sharing \textbf{private-key} (the symmetric-key) in advance.
\item \textbf{Implicit assumption}: there is some way of initially sharing a key in a secret manner.
\item \textbf{Disk encryption}: the same user at different points in time.
\end{itemize}
\end{frame}
\begin{frame}\frametitle{The Syntax of Encryption}
\begin{figure}
\begin{center}
\begin{tikzpicture}
\node (sender) {\Lisa};
\node (bart) [below of = sender] {Alice};
\node (enc) [draw, right of = sender, rounded corners=1ex,node distance = 2cm] {$\mathsf{Enc}$};
\node (k1) [above of = enc, node distance = 1cm] {$k$};
\node (c) [right of = enc, node distance = 2cm] {$c$};
\node (gen) [draw, above of = c, rounded corners=1ex,node distance = 1cm] {$\mathsf{Gen}$};
\node (adv) [below of = c, node distance = 1cm] {\Burns};
\node (burns) [below of = adv] {Adversary};
\node (dec) [draw, right of = c, rounded corners=1ex,node distance = 2cm] {$\mathsf{Dec}$};
\node (k2) [above of = dec, node distance = 1cm] {$k$};
\node (receiver) [right of = dec, node distance = 2cm] {\Left\Bart};
\node (lisa) [below of = receiver] {Bob};
\draw[-latex] (sender) -- (enc) node [midway, above] {$m$};
\draw (enc) -- (c); \draw[-latex] (c) -- (dec);
\draw[-latex] (dec) -- (receiver) node [midway, above] {$m$};
\draw[-latex] (k1) -- (enc);
\draw[-latex] (gen) -- (k1);
\draw[-latex] (gen) -- (k2);								
\draw[-latex] (k2) -- (dec);		
\end{tikzpicture}
\end{center}
\end{figure}
\begin{itemize}
\item key $k \in \mathcal{K}$, plaintext (or message) $m \in \mathcal{M}$, ciphertext $c \in \mathcal{C}$.
\item \textbf{Key-generation} algorithm~$k \gets \mathsf{Gen}$.
\item \textbf{Encryption} algorithm~$c:= \mathsf{Enc}_k(m)$.
\item \textbf{Decryption} algorithm~$m:= \mathsf{Dec}_k(c)$.
\item \textbf{Encryption scheme}: $\Pi = (\mathsf{Gen}, \mathsf{Enc}, \mathsf{Dec})$.
\item \textbf{Basic correctness requirement}: $\mathsf{Dec}_k(\mathsf{Enc}_k(m)) = m$.
\end{itemize}
\end{frame}
\begin{frame}\frametitle{Securing Key vs Obscuring Algorithm}
\begin{itemize}
\item Easier to maintain secrecy of a short key.
\item In case the key is exposed, easier for the honest parties to change the key.
\item In case many pairs of people, easier to use the same algorithm, but different keys.
\end{itemize}
\begin{alertblock}{Kerckhoffs's principle}
\begin{quote}
The cipher method must not be required to be secret, and it must be able to fall into the hands of the enemy without inconvenience.
\end{quote}	
\end{alertblock}
\end{frame}
\begin{frame}\frametitle{Why ``Open Cryptographic Design''}
\begin{itemize}
\item Published designs undergo public scrutiny are to be stronger.
\item Better for security flaws to be revealed by ``ethical hackers''.
\item Reverse engineering of the code (or leakage by industrial espionage) poses a serious threat to security.
\item Enable the establishment of standards.
\end{itemize}
\end{frame}
\begin{frame}\frametitle{Attack Scenarios}	
\begin{itemize}
\item \textbf{Ciphertext-only}: the adversary just observes ciphertext
\item \textbf{Known-plaintext}: the adversary learns pairs of plaintexts/ciphertexts under the same key
\item \textbf{Chosen-plaintext}: the adversary has the ability to obtain the encryption of plaintexts of its choice
\item \textbf{Chosen-ciphertext}: the adversary has the ability to obtain the decryption of \textbf{other} ciphertexts of its choice
\item \textbf{Passive attack}: COA KPA
\begin{itemize}
\item because not all ciphertext are confidential
\end{itemize}
\item \textbf{Active attack}: CPA CCA
\begin{itemize}
\item when to encrypt/decrypt whatever an adversary wishes?
\end{itemize}
\end{itemize}	
\end{frame}
\section{Historical Ciphers and Their Cryptanalysis}
\begin{comment}
	\begin{frame}\frametitle{Why We Learn Broken Ciphers?}
	\begin{itemize}
	\item To understand the weaknesses of an ``ad-hoc'' approach
	\item To learn that ``simple'' approaches are unlikely to succeed
	\item To feel that ``we are smart enough to do some crypt-analyzing''
	\end{itemize}
	\end{frame}
\end{comment}

\begin{frame}[fragile]\frametitle{Caesar's Cipher}
\begin{quote}
If he had anything confidential to say, he wrote it in cipher, that is, by so changing the order of the letters of the alphabet, that not a word could be made out. If anyone wishes to \alert{decipher} these, and get at their meaning, he must \alert{substitute the fourth letter of the alphabet, namely D, for A}, and so with the others

\rightline{--Suetonius,``Life of Julius Caesar''}
\end{quote}
\begin{itemize}
	\item $\mathsf{Enc}(m)=m+3\mod 26$ \footnote{In fact the quote indicates that decryption involved rotating letters of the alphabet forward 3 positions, $\mathsf{Dec}(c)=c+3\mod 26$}
	\item \textbf{Weakness}: \alert{What is the key?}
\end{itemize}
\begin{exampleblock}{Example}
\verb|begintheattacknow|
%\verb|EHJLQWKHDWWDFNQRZ|
\end{exampleblock}
\end{frame}
\begin{frame}[fragile]\frametitle{Shift Cipher}
\begin{itemize}
\item $\mathsf{Enc}_k(m)=m+k\mod 26$
\item $\mathsf{Dec}_k(c)=c-k\mod 26$
\item \textbf{Weakness}: Fragile under \textbf{Brute-force attack} (exhaustive search)
\end{itemize}
\begin{exampleblock}{Example: Decipher the string}	
\verb|EHJLQWKHDWWDFNQRZ|
\end{exampleblock}
\begin{alertblock}{Sufficient Key Space Principle}
Any secure encryption scheme must have a key space that is not vulnerable to exhaustive search.\footnote{If the plaintext space is larger than the key space.}
\end{alertblock}
\end{frame}
\begin{frame}\frametitle{Index of Coincidence (IC) Method (to find $k$)}
\textbf{Index of Coincidence (IC)}: the probability that two randomly selected letters (pick-then-return) will be identical.

Let $p_i$ denote the probability of $i$th letter in English text.
\[I \overset{\text{def}}{=}\sum_{i=0}^{25} p_i^2 \]
\begin{exampleblock}{Example}
What's the IC of `apple'?
\end{exampleblock}

For a long English text, the IC is $\approx 0.065$.
For $j = 0, 1, \dotsc , 25$, $q_j$ is the probability of $j$th letter in the ciphertext.
\[I_j \overset{\text{def}}{=}\sum_{i=0}^{25} p_i \cdot q_{i+j}\]
\alert{Q: For shift cipher, if $j = k$, then $I_j \approx$ ?}

\end{frame}
\begin{frame}[fragile]\frametitle{Mono-Alphabetic Substitution}
\begin{itemize}
\item \textbf{Idea}: To map each character to a different one in an arbitrary manner.
\item \textbf{Strength}: Key space is large $\approx 2^{88}$. \alert{Q: how to count?}
\item \textbf{Weakness}: The mapping of each letter is fixed.
\end{itemize}
\begin{exampleblock}{Example}
\verb|abcdefghijklmnopqrstuvwxyz|\\
\verb|XEUADNBKVMROCQFSYHWGLZIJPT|

Plaintext: \verb|tellhimaboutme|\\
Ciphertext: \verb|??????????????|
\end{exampleblock}
\end{frame}
\begin{frame}[fragile]\frametitle{Attack with Statistical Patterns}
\begin{enumerate}
\item Tabulate the frequency of letters in the ciphertext.
\item Compare it to those in English text.
\item Guess the most frequent letter corresponds to \verb|e|, and so on.
\item Choose the plaintext that does ``make sense''. (Not trivial)
\end{enumerate}
\begin{table}
\begin{center}
\caption{Average letter frequencies for English-language text}
\begin{tabular}{|cc|cc|cc|cc|cc|} \hline
e & 12.7\% & t & 9.1\% & a & 8.2\% & o & 7.5\% & i & 7.0\%\\
n & 6.7\% & \_ & 6.4\% & s & 6.3\% & h & 6.1\% & r & 6.0\%\\
d & 4.3\% & l & 4.0\% & c & 2.8\% & u & 2.8\% & m & 2.4\%\\
w & 2.4\% & f & 2.2\% & g & 2.0\% & y & 2.0\% & p & 1.9\%\\
b & 1.5\% & v & 1.0\% & k & 0.8\% & j & 0.2\% & x & 0.2\%\\
q & 0.1\% & z & 0.1\% & & & & & &\\ \hline
\end{tabular}
\end{center}
\end{table}
\end{frame}
\begin{frame}[fragile]\frametitle{Example of Frequency Analysis (Ciphertext)}
\begin{verbatim}
LIVITCSWPIYVEWHEVSRIQMXLEYVEOIEWHRXEXIPFEMVEWHKVS
TYLXZIXLIKIIXPIJVSZEYPERRGERIMWQLMGLMXQERIWGPSRIH
MXQEREKIETXMJTPRGEVEKEITREWHEXXLEXXMZITWAWSQWXSWE
XTVEPMRXRSJGSTVRIEYVIEXCVMUIMWERGMIWXMJMGCSMWXSJO
MIQXLIVIQIVIXQSVSTWHKPEGARCSXRWIEVSWIIBXVIZMXFSJX
LIKEGAEWHEPSWYSWIWIEVXLISXLIVXLIRGEPIRQIVIIBGIIHM
WYPFLEVHEWHYPSRRFQMXLEPPXLIECCIEVEWGISJKTVWMRLIHY
SPHXLIQIMYLXSJXLIMWRIGXQEROIVFVIZEVAEKPIEWHXEAMWY
EPPXLMWYRMWXSGSWRMHIVEXMSWMGSTPHLEVHPFKPEZINTCMXI
VJSVLMRSCMWMSWVIRCIGXMWYMX
\end{verbatim}
\end{frame}
\begin{frame}[fragile]\frametitle{Example of Frequency Analysis (Analysis)}
Count and Guess, Trial and Error.
\begin{table}
\begin{center}
\caption{Analysis Steps}
\begin{tabular}{|r|l|} \hline
Ciphertext & Plaintext \\ \hline
\alert{I}   & \alert{e} \\
\alert{XLI} & \alert{the} \\
\alert{E} & \alert{a} \\
\alert{R}tate & \alert{s}tate \\
atthatt\alert{MZ}e & atthatt\alert{im}e \\
he\alert{V}e & he\alert{r}e \\
remar\alert{A} & remar\alert{k} \\ \hline
\end{tabular}
\end{center}
\end{table}
\end{frame}
\begin{frame}[fragile]\frametitle{Example of Frequency Analysis (Plaintext)}
\begin{quote}
Hereupon Legrand arose, with a grave and stately air, and brought me the beetle
from a glass case in which it was enclosed. It was a beautiful scarabaeus, and, at
that time, unknown to naturalists -- of course a great prize in a scientific point
of view. There were two round black spots near one extremity of the back, and a
long one near the other. The scales were exceedingly hard and glossy, with all the
appearance of burnished gold. The weight of the insect was very remarkable, and,
taking all things into consideration, I could hardly blame Jupiter for his opinion
respecting it.

\rightline{--Edgar Allan Poe's ``The Gold-Bug''}
\end{quote}
\end{frame}

\begin{frame}[fragile]\frametitle{Vigen\`{e}re (poly-alphabetic shift) Cipher}
\begin{itemize}
\item \textbf{Idea}: To ``smooth out'' the distribution in the ciphertext by mapping different instances of the same letter in the plaintext to different ones in the ciphertext
\item \textbf{Encryption}: $c_i=m_i+k_{[i\bmod t]}$, $t$ is the length (period) of $k$
\item \textbf{Cryptanalysis}: Need find $t$; if $t$ is known, need know whether the decryption ``makes sense'', but brute force ($26^t$) is infeasible for $t > 15$
\end{itemize}
\begin{exampleblock}{Example (Key is `cafe')}
\begin{description}[Ciphertext]
\item[Plaintext]  \verb|tellhimaboutme| \\
\item[Key]        \verb|cafecafecafeca| \\
\item[Ciphertext] \verb|??????????????| %\verb|WFRQKJSFEPAYPF|
\end{description}
\end{exampleblock}
\end{frame}
\begin{frame}[fragile]\frametitle{Kasiski's Method (to find $t$)}
\begin{itemize}
\item To identify repeated patterns of length 2 or 3.
\item The distance between such appearances is a multiple of $t$.
\item $t$ is the greatest common divisor of all the distances.
\end{itemize}
\begin{exampleblock}{Example (Key is `beads')}
\begin{semiverbatim}
themanandthewomanretrievedtheletterfromthepostoffice
beadsbeadsbeadsbeadsbeadsbeansdeadsbeadsbeadsbeadbea
VMFQTPFOH\alert{MJJ}XSFCSSIMTNFZXFYISEIYUIKHWPQ\alert{MJJ}QSLVTGJKGF
\end{semiverbatim}
\end{exampleblock}
\end{frame}
\begin{frame}\frametitle{Index of Coincidence (IC) Method (to find $t$)}
For $\tau = 1, 2, \dotsc$, $q_i$ is the probability of $i$th letter in $c_1, c_{1+\tau}, c_{1+2\tau}, \dotsc$, IC is
\[I_\tau \overset{\text{def}}{=}\sum_{i=0}^{25} q_i^2\]
\alert{If $\tau = t$, then $I_\tau \approx ?$} ; otherwise $q_i \approx \frac{1}{26}$ and
\[I_\tau \approx \sum_{i=0}^{25} \left(\frac{1}{26}\right)^2 \approx 0.038\]
Then reuse IC method to find $k_i$.
\begin{alertblock}{Arbitrary Adversary Principle}
Security must be guaranteed for any adversary within the class of adversaries having the specified power
\end{alertblock}
\end{frame}
\begin{frame}\frametitle{Cryptanalytic Attacks (homework assignment)}
\begin{itemize}
\item Under COA, the requirement for ciphertext related to the size of the key space.  Vig\`{e}nere > mono-alphabetic sub. > shift
\item Under KPA, trivially broken.
\end{itemize}
\begin{alertblock}{Lessons learned}
\begin{itemize}
\item Sufficient key space principle.
\item Designing secure cipher is a hard task.
\item Complexity does not imply security. (then what does?)
\item Arbitrary adversary principle
\end{itemize}
\end{alertblock}
\end{frame}
\section{The Basic Principles of Modern Cryptography}
\begin{frame}\frametitle{Three Main Principles of Modern Cryptography}
\begin{enumerate}
\item The formulation of a rigorous \textbf{definition} of security / threat model.
\item When the security of a cipher relies on an unproven \textbf{assumption}, this assumption must be precisely stated and be as minimal as possible.
\item Cipher should be accompanied by a rigorous \textbf{proof} of security with the above definition and the above assumption.
\end{enumerate}
\end{frame}
\begin{frame}\frametitle{Why Principle 1 -- Formulation of Exact Definitions}
\begin{exampleblock}{Q: how would you formalize the security for private-key encryption?}
\begin{enumerate}
\item \emph{No adversary can find the secret key when given a ciphertext.}\\
$\mathsf{Enc}_k(m)=m$
\item \emph{No adversary can find the plaintext that corresponds to the ciphertext.}\\
$\mathsf{Enc}_k(m)=m_{0}\| \mathsf{AES}_k(m)$
\item \emph{No adversary can determine any character of the plaintext that corresponds to the ciphertext.}\\
$m=1000$, someone can learn $ 800 < m < 1200$
\item \emph{No adversary can derive any meaningful information about the plaintext from the ciphertext.}\\
Could you define so-called `meaningful'?
\end{enumerate}
\emph{\alert{Definitions of security should suffice for all potential applications.}}
\end{exampleblock}
\end{frame}
\begin{frame}\frametitle{Why Principle 1 -- How to define}
%\begin{exampleblock}{General Form}
%A cryptographic scheme for a given \textbf{task} is secure if no adversary of a specified \textbf{power} can achieve a specified \textbf{break}
%\end{exampleblock}

How To Define Security -- Lesson From Alan Turing
\begin{itemize}
\item What's computation?\footnote{Q: Any ``mathematical proof that there exist well-defined problems that computers cannot solve''? A: Halting Problem in computability theory}
\begin{enumerate}
\item A direct appeal to \textbf{intuition}
\item A \textbf{proof of the equivalence} of two definitions\\ (The new one has a greater intuitive appeal)
\item Giving \textbf{examples} solved using a definition
\end{enumerate}
\item Additional method for security: \textbf{Test of time}
\end{itemize}
\end{frame}	
\begin{frame}\frametitle{Principle 2 -- Reliance on Precise Assumptions}
Most cryptographic constructions \textbf{cannot be proven secure unconditionally}
\begin{itemize}
	\item \textbf{Why?} 
	\begin{enumerate}
		\item Validation of the assumption
		\item Comparison of schemes
		\item Facilitation of proofs of security
	\end{enumerate}
	\textbf{The construction is secure if the assumption is true.}
	\item \textbf{How?} 
	\begin{enumerate}
		\item old, so well tested
		\item simple and lower-level, so easy to study, refute \& correct
	\end{enumerate}
\end{itemize}
\end{frame}
\begin{frame}\frametitle{Principle 3 -- Rigorous Proofs of Security}
\begin{itemize}
\item \textbf{Why?} Proofs are more desirable in computer security than in other fields.
\item \textbf{The reductionist approach}: 
\begin{theorem}	Given that Assumption X is true, Construction Y is secure according to the given definition.
\end{theorem}
\begin{proof} Reduce the problem given by X to the problem of breaking Y.
\end{proof}
\item \textbf{Ad-hoc approaches}: for those who need a ``quick and dirty'' solution, or who are just simply unaware.
\end{itemize}
\end{frame}
\begin{frame}\frametitle{Summary}
\begin{itemize}
\item Cryptography secures information, transactions and computations
\item Kerckhoffs's principle \& Open cryptographic design
\item Caesar's, shift, Mono-Alphabetic sub., Vigen\`{e}re
\item Brute force, letter frequency, Kasiski's, IC
\item Sufficient key space principle
\item Arbitrary adversary principle
\item Rigorously proven security
\end{itemize}
\end{frame}
\end{document}


%% presentation
\documentclass{beamer}
\usetheme[height=7mm]{Rochester}
\usecolortheme{rose}

% handout

%\documentclass[handout]{beamer}
%\usepackage{pgfpages} \pgfpagesuselayout{8 on 1}[a4paper]

%\documentclass[mathserif]{article}
%\usepackage{beamerarticle}

\usepackage{amsmath}
\usepackage{comment}
\usepackage{amssymb,amsfonts}
\usepackage[T1]{fontenc}
\usepackage{lmodern}
\usepackage{tikz}
\usepackage{simpsons}
\usepackage{marvosym}
\usepackage{color}
\usepackage{multirow}
\usepackage{pgffor}
\usepackage[slide,algoruled,titlenumbered,vlined,noend,linesnumbered,]{algorithm2e}

\usefonttheme{structurebold}

\setbeamertemplate{footline}[frame number]
\setbeamertemplate{navigation symbols}{}
\setbeamerfont{smallverb}{size*={73}}
\usefonttheme[onlymath]{serif}
\setbeamertemplate{theorems}[numbered]
\newtheorem{construction}[theorem]{Construction}
\newtheorem{proposition}[theorem]{Proposition}

\AtBeginSection[] { 
  \begin{frame} 
    \frametitle{Content} 
    \tableofcontents[currentsection]
  \end{frame} 
  \addtocounter{framenumber}{-1} 
}

\usetikzlibrary[shapes.arrows]
\usetikzlibrary{shapes.geometric}
\usetikzlibrary{backgrounds}
\usetikzlibrary{positioning}
\usetikzlibrary{calc}
\usetikzlibrary{intersections}
\usetikzlibrary{fadings}
\usetikzlibrary{decorations.footprints}
\usetikzlibrary{patterns}
\usetikzlibrary{shapes.callouts}
\usetikzlibrary{fit}
%handout

\providecommand{\abs}[1]{\lvert#1\rvert}

\tikzset{every picture/.style={line width=1pt,show background rectangle},background rectangle/.style={fill=blue!10,rounded corners=2ex}}

\author{Yu Zhang}
\institute{Harbin Institute of Technology}
\date[Crypt'16A]{Cryptography, Autumn, 2016}

%\input{1introduction.tex}
%\input{2perfectlysecret.tex}
%\input{3privatekey.tex}


\title{Perfectly Secret Encryption}

\begin{document}
\maketitle
\begin{frame}\frametitle{Outline}
\tableofcontents
\end{frame}
\section{Definitions and Basic Properties}
\begin{frame}\frametitle{Recall The Syntax of Encryption}
\begin{figure}
\begin{center}
\begin{tikzpicture}
\node (sender) {\Lisa};
\node (bart) [below of = sender] {Alice};
\node (enc) [draw, right of = sender, rounded corners=1ex,node distance = 2cm] {$\mathsf{Enc}$};
\node (k1) [above of = enc, node distance = 1cm] {$k$};
\node (c) [right of = enc, node distance = 2cm] {$c$};
\node (gen) [draw, above of = c, rounded corners=1ex,node distance = 1cm] {$\mathsf{Gen}$};
\node (adv) [below of = c, node distance = 1cm] {\Burns};
\node (burns) [below of = adv] {Adversary};
\node (dec) [draw, right of = c, rounded corners=1ex,node distance = 2cm] {$\mathsf{Dec}$};
\node (k2) [above of = dec, node distance = 1cm] {$k$};
\node (receiver) [right of = dec, node distance = 2cm] {\Left\Bart};
\node (lisa) [below of = receiver] {Bob};
\draw[-latex] (sender) -- (enc) node [midway, above] {$m$};
\draw (enc) -- (c); \draw[-latex] (c) -- (dec);
\draw[-latex] (dec) -- (receiver) node [midway, above] {$m$};
\draw[-latex] (k1) -- (enc);
\draw[-latex] (gen) -- (k1);
\draw[-latex] (gen) -- (k2);								
\draw[-latex] (k2) -- (dec);		
\end{tikzpicture}
\end{center}
\end{figure}
\begin{itemize}
\item $k \in \mathcal{K}, m \in \mathcal{M}, c \in \mathcal{C}$.
\item $k \gets \mathsf{Gen}, c:= \mathsf{Enc}_k(m), m:= \mathsf{Dec}_k(c)$.
\item \textbf{Encryption scheme}: $\Pi = (\mathsf{Gen}, \mathsf{Enc}, \mathsf{Dec})$.
\item \textbf{Random Variable}: $K, M, C$ for key, plaintext, ciphertext.
\item \textbf{Probability}: $\Pr[K=k], \Pr[M=m], \Pr[C=c].$
\item \alert{What's the basic correctness requirement?}
\end{itemize}
\end{frame}
\begin{frame}\frametitle{Definition of `Perfect Secrecy'}
\textbf{Intuition}: An adversary knows the probability distribution over $\mathcal{M}$. $c$ should have no effect on the knowledge of the adversary; the a \emph{posteriori} likelihood that some $m$ was sent should be no different from the a \emph{priori} probability that $m$ would be sent. 
\begin{definition}
$\Pi$ over $\mathcal{M}$ is \textbf{perfectly secret} if for every probability distribution over $\mathcal{M}$, $\forall m \in \mathcal{M}$ and $\forall c \in \mathcal{C}$ for which $\Pr[C = c] > 0$:
\[ \Pr[M=m | C=c] = \Pr[M=m].\]
\end{definition}
\textbf{Simplify}: non-zero probabilities for $\forall m \in \mathcal{M}$ and $\forall c \in \mathcal{C}$.\\

\begin{exampleblock}{Is the below scheme perfectly secret?}{ For $\mathcal{M}=\mathcal{K} = \{ 0,1 \} , \mathsf{Enc}_k(m)= m \oplus k$.}\end{exampleblock}
\end{frame}

\begin{frame}\frametitle{An Equivalent Formulation}
\begin{lemma} \label{lem:ps} 
$\Pi$ over $\mathcal{M}$ is perfectly secret $\iff$ for every probability distribution over $\mathcal{M}$, $\forall m \in \mathcal{M}$ and $\forall c \in \mathcal{C}$:
\[ \Pr[C=c | M=m] = \Pr[C=c].\]
\end{lemma}
\begin{proof}
$\Leftarrow$: Multiplying both sides by $\Pr[M=m]/\Pr[C=c]$, then use Bayes' Theorem.\footnote{If $\Pr[B]\neq 0$ then $ \Pr[A|B] = \left( \Pr[A] \cdot \Pr[B|A] \right) / \Pr[B] $} \\
$\Rightarrow$: Multiplying both sides by $\Pr[C=c]/\Pr[M=m]$, then use Bayes' Theorem.
\end{proof}
\end{frame}
\begin{frame}\frametitle{Perfect Indistinguishability}
\begin{lemma}\label{lem:pi}
$\Pi$ over $\mathcal{M}$ is perfectly secret $\iff$ for every probability distribution over $\mathcal{M}$, $\forall m_0, m_1 \in \mathcal{M}$ and $\forall c \in \mathcal{C}$:
\[ \Pr[C=c | M=m_0] = \Pr[C=c | M=m_1].\]
\end{lemma}
\begin{proof}
$\Rightarrow$: By Lemma \ref{lem:ps}: $\Pr[C=c | M=m] = \Pr[C=c]$. \\
$\Leftarrow$: $p \overset{\text{def}}{=} \Pr[C=c | M=m_0]$.
\[
\begin{split}
	\Pr[C=c] &= \sum_{m \in \mathcal{M}} \Pr[C=c|M=m] \cdot \Pr[M=m] \\
	&= \sum_{m \in \mathcal{M}} p \cdot \Pr[M=m] = p = \Pr[C=c|M=m_0].
\end{split}
\]
\end{proof}
\end{frame}
\section{The One-Time Pad (Vernam's Cipher)}
\begin{frame}\frametitle{One-Time Pad (Vernam's Cipher)}
\begin{itemize}
	\item $\mathcal{M} = \mathcal{K} = \mathcal{C} = \{0,1\}^{\ell}$.
	\item $\mathsf{Gen}$ chooses a $k$ randomly with probability exactly $2^{-\ell}$.
	\item $c := \mathsf{Enc}_k(m) = k \oplus m$. 
	\item $m := \mathsf{Dec}_k(c) = k \oplus c$. 
\end{itemize}
\begin{theorem}
The one-time pad encryption scheme is perfectly-secret.
\end{theorem}
\begin{proof}
\[\begin{split} \Pr[C=c|M=m] &= \Pr[M \oplus K=c|M=m] \\
&= \Pr[m \oplus K=c] = \Pr[K = m \oplus c] = 2^{-\ell}.
\end{split}
\]
Then Lemma \ref{lem:pi}: $\Pr[C=c | M=m_0] = \Pr[C=c | M=m_1]$.
\end{proof}
\end{frame}
\section{Limitations of Perfect Secrecy}
\begin{frame}\frametitle{Limitations of OTP and Perfect Secrecy}
Key $k$ is as long as $m$, difficult to store and share $k$.
\begin{theorem}
Let $\Pi$ be perfectly-secret over $\mathcal{M}$, and let $\mathcal{K}$ be determined by $\mathsf{Gen}$. Then $|\mathcal{K}|\ge |\mathcal{M}|$. 
\end{theorem}
\begin{proof}
Assume $|\mathcal{K}| < |\mathcal{M}|$.
$\mathcal{M}(c) \overset{\text{def}}{=} \{ \hat{m} | \hat{m} = \mathsf{Dec}_k(c)\  \text{for some}\ \hat{k} \in \mathcal{K} \}$, and $|\mathcal{M}(c)|\le |\mathcal{K}| < |\mathcal{M}|$. So $\exists m' \notin \mathcal{M}(c)$. Then
\[ \Pr[M=m'|C=c] = 0 \neq \Pr[M = m'] \]
and so not perfectly secret.
\end{proof}
\end{frame}
\begin{frame}\frametitle{Two Time Pad: Real World Cases}
Only used once for the same key, otherwise
\[c\oplus c'=(m\oplus k)\oplus (m'\oplus k)=m\oplus m'.\]
Learn $m$ from $m\oplus m'$ due to the redundancy of language.
\begin{exampleblock}{MS-PPTP (Win NT)}
\begin{figure}
\begin{center}
\begin{tikzpicture}
\node [label=below:Client, label=above:$k$] (sender) {\Lisa};
\node (c) at ($(sender)+(4cm,0.5cm)$) {$\left[ m_1\|m_2\|m_3\right] \oplus PRG(k)$};
\node (c1) [below of = c, node distance = 1cm] {$\left[s_1\|s_2\|s_3\right] \oplus PRG(k)$};
\node (receiver) at ($(sender)+(8cm,0)$) [label=below:Server, label=above:$k$] {\Left\Bart};
\draw[-latex] (sender.east |- c) -- (c) -- (receiver.west |- c);
\draw[-latex] (receiver.west |- c1) -- (c1) -- (sender.east |- c1);
\end{tikzpicture}
\end{center}
\end{figure}
Improvement: use two keys for C-to-S and S-to-C separately.
\end{exampleblock}
\end{frame}
\section{Shannon's Theorem}
\begin{frame}\frametitle{Shannon's Theorem}
\begin{theorem}
For $|\mathcal{M}| = |\mathcal{K}| = |\mathcal{C}|$, $\Pi$ is perfectly secret $\iff$
\begin{enumerate}
\item Every $k \in \mathcal{K}$ is chosen with probability $1/|\mathcal{K}|$ by $\mathsf{Gen}$.
\item $\forall m \in \mathcal{M}$ and $\forall c \in \mathcal{C}$, $\exists$ unique $k \in \mathcal{K}$: $c := \mathsf{Enc}_k(m)$.
\end{enumerate}
\end{theorem}
\begin{proof}
$\Leftarrow$: $\Pr[C=c|M=m]=1/|\mathcal{K}|$, use Lemma \ref{lem:pi}. \\
$\Rightarrow (2)$: At least one $k$, otherwise $\Pr[C=c|M=m]=0$; \\
at most one $k$, because $\{\mathsf{Enc}_k(m)\}_{k\in \mathcal{K}} = \mathcal{C}$ and $|\mathcal{K}| = |\mathcal{C}|$.\\
$\Rightarrow (1)$: $k_i$ is such that $\mathsf{Enc}_{k_i}(m_i)=c$.
\[ \begin{split}
\Pr[M = m_i] &= \Pr[M=m_i|C=c] \\
             &= \left( \Pr[C =c|M=m_i] \cdot \Pr[M = m_i] \right) / \Pr[C=c] \\
 &= \left( \Pr[K=k_i] \cdot \Pr[M = m_i] \right) / \Pr[C=c],
\end{split}
\]
so $\Pr[K=k_i] = \Pr[C = c] = 1/|\mathcal{K}|$.
\end{proof}
\end{frame}

\begin{frame}\frametitle{Application of Shannon's Theorem}
\begin{exampleblock}{Is the below scheme perfectly secret?}
Let $\mathcal{M} = \mathcal{C} = \mathcal{K} = \{ 0, 1, 2,\dots , 255 \} $\\
$\mathsf{Enc}_k(m) = m  + k \mod 256$\\
$\mathsf{Dec}_k(c) = c - k \mod 256$
\end{exampleblock}
\end{frame}
\section{Eavesdropping Indistinguishability}
\begin{frame}\frametitle{Eavesdropping Indistinguishability Experiment}
$\mathsf{PrivK}^{\mathsf{eav}}_{\mathcal{A},\Pi}$ denote a \textbf{priv}ate-\textbf{k}ey encryption experiment for a given $\Pi$ over $\mathcal{M}$ and an \textbf{eav}esdropping adversary $\mathcal{A}$.
\begin{enumerate}
	\item $\mathcal{A}$ outputs a pair of messages $m_0, m_1 \in \mathcal{M}$.
	\item $k \gets \mathsf{Gen}$, a random bit $b \gets \{0,1\}$ is chosen. Then $c \gets \mathsf{Enc}_k(m_b)$ is given to $\mathcal{A}$.
	\item $\mathcal{A}$ outputs a bit $b'$
	\item If $b' = b$, $\mathcal{A}$ succeeded $\mathsf{PrivK}^{\mathsf{eav}}_{\mathcal{A},\Pi}=1$, otherwise 0.
\end{enumerate}
\begin{figure}
\begin{center}
\begin{tikzpicture}
\node (A) at (0,0) {\Homer};
\node (B) [right of = A, node distance = 4cm] {\Left\Burns};
\node (1a) [below of=A, node distance=1cm] {};
\node (1b) [below of=B, node distance=1cm] {$m_0, m_1$};
\draw[-latex] (1b) -- (1a) node [midway,above] {};
\node (2a) [below of=1a, node distance=0.5cm] {Gen $b, k$};
\node (2b) [below of=1b, node distance=0.5cm] {};
%\draw[-latex] (2b) -- (2a) node [midway,above] {};
%\node (3a) [below of=2a, node distance=0.5cm] {};
%\node (3b) [below of=2b, node distance=0.5cm] {};
\node (4a) [below of=2a, node distance=0.5cm] {$\mathsf{Enc}_k(m_b)$};
\node (4b) [below of=2b, node distance=0.5cm] {};
\draw[-latex] (4a) -- (4b) node [midway,above] {};
\node (5a) [below of=4a, node distance=0.5cm] {};
\node (5b) [below of=4b, node distance=0.5cm] {$b'$};
\draw[-latex] (5b) -- (5a) node [midway,above] {};
\node (6a) [below of=5a, node distance=0.5cm] {};
\node (6b) [below of=5b, node distance=0.5cm] {};
\node (result) [right of = 6a, node distance = 2cm] {Win if $b = b'$};
\end{tikzpicture}

\end{center}
\end{figure}
\end{frame}
\begin{frame}\frametitle{Adversarial Indistinguishability}
\begin{definition}
$\Pi$ over $\mathcal{M}$ is \textbf{perfectly secret} if for every $\mathcal{A}$ it holds that
\[ \Pr[\mathsf{PrivK}^{\mathsf{eav}}_{\mathcal{A},\Pi}=1] = \frac{1}{2}.\]
\end{definition}
\begin{exampleblock}{Which in the below schemes are perfectly secret?}
\begin{itemize}
\item $\mathsf{Enc}_{k,k'}(m)= \mathsf{OTP}_k(m) \| \mathsf{OTP}_{k'}(m)$
\item $\mathsf{Enc}_{k}(m)= reverse(\mathsf{OTP}_k(m))$
\item $\mathsf{Enc}_{k}(m)= \mathsf{OTP}_k(m) \| k$
%To break semantic security, an attacker would read the secret key from the challenge ciphertext and use it to decrypt the challenge ciphertext. Basically, any ciphertext reveals the secret key.
\item $\mathsf{Enc}_{k}(m)= \mathsf{OTP}_k(m) \| \mathsf{OTP}_k(m) $
\item $\mathsf{Enc}_{k}(m)= \mathsf{OTP}_{0^{n}}(m)$
%To break semantic security, an attacker would ask for the encryption of $0^n$ and $1^n$ and can easily distinguish EXP(0) from EXP(1) because it knows the secret key, namely 0n.
\item $\mathsf{Enc}_{k}(m)= \mathsf{OTP}_k(m) \| LSB(m)$
%To break semantic security, an attacker would ask for the encryption of $0^n$ and $0^{n-1}1$ and can distinguish EXP(0) from EXP(1).
\end{itemize}
\end{exampleblock}
\end{frame}

\begin{frame}\frametitle{Summary}
\begin{itemize}
\item Perfect secrecy $=$ Perfect indistinguishability $=$ Adversarial indistinguishability.
\item Perfect secrecy is attainable. The One-Time Pad (Vernam's cipher).
\item Shannon's theorem.
\end{itemize}	
\end{frame}
\end{document}

%\input{3privatekey.tex}


\title{Number Theory and RSA Problem}

\begin{document}
\maketitle
\begin{frame}
\frametitle{Outline}
\tableofcontents
\end{frame}
%\begin{frame}\frametitle{The World of Private-Key Cryptography}
%\begin{figure}
%\begin{center}
%\begin{tikzpicture}[block/.style={rectangle, draw,align=center, rounded corners,
minimum height=2em, minimum width=11em}]
\node (pke) [block] {Private-Key Encryption};
\node (mac) [right of=pke,block,node distance=5cm] {Message Authentication Codes};
\node (prf) at ($(pke)+(2.5cm,-1.5cm)$) [block] {Pseudorandom Functions};
\node (bc) [below of=pke,block,node distance=3cm] {Block Ciphers};
\node (prg) [below of=mac,block,node distance=3cm] {Pseudorandom Generators};
\node (owf) [below of=prg,block,node distance=1.5cm] {One-Way Function};
\node (rsa) [below of=owf,block,node distance=1.5cm] {RSA, Discrete Log, Factoring};
\draw[-latex] (prf) -- (pke);
\draw[-latex] (prf) -- (mac);
\draw[-latex] (bc) -- (prf);
\draw[-latex] (prg) -- (prf);
\draw[-latex] (owf) -- (prg);
\draw[-latex] (rsa) -- (owf);
\end{tikzpicture}
%\end{center}
%\end{figure}
%\end{frame}
\section{Arithmetic and Basic Group Theory}
\begin{frame}\frametitle{Primes and Divisibility}
\begin{itemize}
\item The set of \textbf{integers} $\mathbb{Z}$, $a,b,c \in \mathbb{Z}$.
\item $a$ \textbf{divides} $b$: $a \mid b$ if $\exists c, ac=b$ (otherwise $a \nmid b$). \\$b$ is a \textbf{multiple} of $a$. If $a \notin \{1,b\}$, then $a$ is a \textbf{factor} of $b$. 
\item $p > 1$ is \textbf{prime} if it has no factors.
\item An integer $>1$ which is not prime is \textbf{composite}.
\item $\forall a,b$, $\exists$ \textbf{quotient} $q$, \textbf{remainder} $r$: $a=qb+r$, and $0\le r < b$.
\item \textbf{Greatest common divisor} $\gcd(a,b)$ is the largest integer $c$ such that $c\mid a$ and $c\mid b$. $\gcd(0,b)=b$, $\gcd(0,0)$ undefined.
\item $a$ and $b$ are \textbf{relatively prime (coprime)} if $\gcd(a,b)=1$.
\item \textbf{Euclid's theorem}: there are infinitely many prime numbers.
\end{itemize}
\end{frame}
\begin{frame}\frametitle{Fundamental Theorem of Arithmetic}
\begin{itemize}
\item \textbf{B\'{e}zout's lemma}: $\forall a,b,\;\exists\;X,Y:\;Xa+Yb=\gcd(a,b)$. $\gcd(a,b)$ is the smallest positive integer that can be expressed in this way.
\item \textbf{Euclid's lemma}: If $c \mid ab$ and $\gcd(a,c)=1$, then $c \mid b$. \\
If $p$ is prime and $p\mid ab$, then either $p \mid a$ or $p \mid b$.
\item \textbf{Fundamental theorem of arithmetic}: $\forall N >1$, $N = \prod _i p_i^{e_i}$, $\{p_i\}$ are distinct primes and $e_i \ge 1$. This expression is unique.
\end{itemize}
\end{frame}
\begin{frame}\frametitle{Modular Arithmetic}
\begin{itemize}
\item Remainder $r= [a\bmod N] = a - b\lfloor a/b\rfloor $  and $r<N$. $N$ is called \textbf{modulus}.
\item \textbf{Reduction modulo} $N$: mapping $a$ to $[a \bmod N]$.
\item $\mathbb{Z}_N = \{0,1,\dots,N-1\} = \{a \bmod N | a \in \mathbb{Z}\}$.
\item $a$ and $b$ are \textbf{congruent modulo} $N$: $a \equiv b \pmod N$ if $[a \bmod N] = [b \bmod N]$.
\item $a$ is \textbf{invertible modulo} $N$ $\iff \gcd(a,N) = 1$. If $ab \equiv 1 \pmod N$, then $b=a^{-1}$ is \textbf{multiple inverse} of $a$ \textbf{modulo} $N$.
\item \textbf{Cancellation law}: If $\gcd(a,N)=1$ and $ab \equiv ac \pmod N$, then $b \equiv c \pmod N$.
\item \textbf{Euclidean algorithm}: $\gcd(a,b) = \gcd(b, [a \bmod b]).$
\item \textbf{Extended Euclidean algorithm}: Given $a,N$, find $X,Y$ with $Xa+YN = \gcd(a,N)$.
\end{itemize}
\end{frame}
\begin{frame}\frametitle{Examples of Modular Arithmetic}
``Reduce and then add/multiply'' instead of ``add/multiply and then reduce''.
\begin{exampleblock}{Compute $193028 \cdot 190301 \bmod 100$}
$193028 \cdot 190301 = [193028 \bmod 100] \cdot [190301 \bmod 100] \bmod 100$
$= 28\cdot 1 \equiv 28 \bmod 100.$
\end{exampleblock}
$ab \equiv cb \pmod N$ does \emph{not necessarily} imply $a \equiv c \pmod N$.
\begin{exampleblock}{$a=3, c=15, b=2, N=24$}
$3\cdot 2 = 6 \equiv 15 \cdot 2 \pmod{24}$, but $3 \not \equiv 15 \pmod{24}$.
\end{exampleblock}
Use extended Euclidean algorithm to ...
\begin{exampleblock}{Find the inverse of $11 \pmod {17}$}
$(-3)\cdot 11 + 2\cdot 17 = 1$, so 14 is the inverse of 11.
\end{exampleblock}
\end{frame}
\begin{frame}\frametitle{Groups}
A \textbf{group} is a set $\mathbb{G}$ with a binary operation $\circ$:
\begin{itemize}
\item (\textbf{Closure}:) $\forall g,h \in \mathbb{G}$, $g \circ h \in \mathbb{G}$.
\item (\textbf{Existence of an Identity}:) $\exists$ \textbf{identity} $e\in \mathbb{G}$ such that $\forall g\in \mathbb{G}, e \circ g = g = g \circ e$.
\item (\textbf{Existence of Inverses}:) $\forall g \in G$, $\exists\; h \in \mathbb{G}$ such that $g \circ h =e = h \circ g$. $h$ is an \textbf{inverse} of $g$.
\item (\textbf{Associativity}:) $\forall g_1,g_2,g_3 \in \mathbb{G}$, $(g_1\circ g_2)\circ g_3 = g_1 \circ (g_2 \circ g_3)$.
\end{itemize}
$\mathbb{G}$ with $\circ$ is \textbf{abelian} if
\begin{itemize}
\item (\textbf{Commutativity}:) $\forall g,h \in \mathbb{G}, g\circ h = h\circ g$.
\end{itemize}

Existence of inverses implies \textbf{cancellation law}.\\
When $\mathbb{G}$ is a \textbf{finite group} and $\abs{\mathbb{G}}$ is the \textbf{order} of group.\\
\end{frame}
\begin{frame}\frametitle{Group Exponentiation}
%\[mg = m\cdot g \overset{\text{def}}{=} \underbrace{g+\cdots +g}_{m\; \text{times}}.\]
\[ g^m \overset{\text{def}}{=} \underbrace{g\circ g\circ \cdots \circ g}_{m\; \text{times}}. \]
\begin{theorem}
$\mathbb{G}$ is a finite group. Then $\forall g \in \mathbb{G}, g^{\abs{\mathbb{G}}}=1$.
\end{theorem}
\begin{corollary}
$\forall g \in \mathbb{G}$ and $i$, $g^i = g^{[i \bmod {\abs{\mathbb{G}}}]}$.
\end{corollary}
\begin{corollary}
Define function $f_e\;:$ $\mathbb{G} \to \mathbb{G}$ by $f_e(g) =g^e$. \\
If $\gcd(e,\abs{\mathbb{G}})=1$, then $f_e$ is a permutation. \\
Let $d = [e^{-1} \bmod {\abs{\mathbb{G}}}]$, then $f_d$ is the inverse of $f_e$. ($f_d(f_e(g))=g$)\\
\textbf{$e$'th root of $c$}: $g^e = c$, $g = c^{1/e} = c^{d}$. 
\end{corollary}
\end{frame}
\begin{frame}\frametitle{The Group $\mathbb{Z}_N^*$}
\[ \mathbb{Z}_N^* \overset{\text{def}}{=} \{a \in \{1,\dotsc,N-1 \} | \gcd(a,N) = 1\} \]
\textbf{Euler's phi function}: $\phi(N) \overset{\text{def}}{=} \abs{\mathbb{Z}_N^*}$.
\begin{theorem}
$N = \prod_ip_i^{e_i}$, $\{p_i\}$ are distinct primes, $\phi(N) = \prod_ip_i^{e_i-1}(p_i-1)$.
\end{theorem}
\begin{corollary}[Euler's theorem \& Fermat's little theorem]
$a \in \mathbb{Z}_N^*$. $a^{\phi (N)} \equiv 1 \pmod N$.\\
If $p$ is prime and $a \in \{1,\dotsc,p-1\}$, then $a^{p-1} \equiv 1 \pmod p$.
\end{corollary}
\begin{corollary}
Define function $f_e\;:$ $\mathbb{Z}^*_N \to \mathbb{Z}^*_N$ by $f_e(x) =[x^e \bmod N]$. \\ If $\gcd(e,\phi(N))=1$, then $f_e$ is a permutation. \\
Let $d = [e^{-1} \bmod \phi(N)]$, then $f_d$ is the inverse of $f_e$.\\
\textbf{$e$'th root of $c$}: $g^e = c$, $g = c^{1/e} = c^{d}$. 
\end{corollary}
\end{frame}
\begin{comment}
\begin{frame}\frametitle{Subgroups}
If $\mathbb{G}$ is a group, a set $\mathbb{H} \subseteq \mathbb{G}$ is a \textbf{subgroup} of $\mathbb{G}$ if $\mathbb{H}$ itself forms a group under the same operation associated with $\mathbb{G}$. $\mathbb{H}$ is a \textbf{strict subgroup} if $\mathbb{H} \neq \mathbb{G}$.
\begin{itemize}
\item If $\mathbb{H} \subseteq \mathbb{G}$, $\mathbb{H}$ contains the identity element of $\mathbb{G}$, and $\mathbb{H}$ is closed, then $\mathbb{H}$ is a subgroup of $\mathbb{G}$.
\item \textbf{Lagrange's theorem}: For a finite group $\mathbb{G}$ and its subgroup $\mathbb{H}$,  $\abs{\mathbb{H}} \mid \abs{\mathbb{G}}$.
\item $\mathbb{H}$ is a strict subgroup of a finite group $\mathbb{G}$, then $\abs{\mathbb{H}} \le \abs{\mathbb{G}}/2$.
\end{itemize}
\end{frame}
\end{comment}
\begin{frame}\frametitle{Examples on Groups}
\begin{exampleblock}{}
\begin{itemize}
\item $\mathbb{Z}$ is an abelian group under `$+$', not a group under `$\cdot$'.
\item The set of real numbers $\mathbb{R}$ is not a group under `$\cdot$'.
\item $\mathbb{R}\setminus \{0\}$ is an abelian group under `$\cdot$'.
\item $\mathbb{Z}_N$ is an abelian group under `$+$' modulo $N$.
\item If $p$ is prime, then $\mathbb{Z}_p^*$ is an abelian group under `$\cdot$' modulo $p$. 
\item $\mathbb{Z}_{15}^*= \{1,2,4,7,8,11,13,14\}$, $\abs{\mathbb{Z}_{15}^*}=8$.
\item $\mathbb{Z}_{3}^*$ is a subgroup of $\mathbb{Z}_{15}^*$, but $\mathbb{Z}_{5}^*$ is not.
\item $2^{1/3} \bmod 5 = 2^{3} \bmod 5 = 3$. ($3^{-1} = 3 \pmod 4$)
\item $g^3$ is a permutation on $\mathbb{Z}_{15}^*$, but $g^2$ is not (e.g., $8^2 \equiv 2^2\equiv 4$). 
\end{itemize}
\end{exampleblock}
\begin{exampleblock}{$N=pq$ where $p,q$ are distinct primes. $\phi(N)=?$}
$\phi(N)=(N-1)-(q-1)-(p-1)=(p-1)(q-1)$.
\end{exampleblock}
\end{frame}
\begin{comment}
\begin{frame}\frametitle{Isomorphism and Cross Product}
A bijection function $f : \mathbb{G} \to \mathbb{H}$ is an \textbf{isomorphism from} $\mathbb{G}$ \textbf{to} $\mathbb{H}$:
\[ \forall g_1,g_2 \in \mathbb{G}, f(g_1 \circ_{\mathbb{G}} g_2) = f(g_1) \circ_{\mathbb{H}} f(g_2).\]
If $\exists$ such $f$, $\mathbb{G} \simeq \mathbb{H}$.\newline

The \textbf{cross product} of $\mathbb{G}$ and $\mathbb{H}$: $\mathbb{G} \times \mathbb{H}$. The elements are $(g,h)$ with $g \in \mathbb{G}$ and $h \in \mathbb{H}$, the operation $\circ$,
\[ (g,h)\circ (g',h') \overset{\text{def}}{=} (g \circ_{\mathbb{G}} g', h \circ_{\mathbb{H}} h')\]
\end{frame}
\begin{frame}\frametitle{Chinese Remainder Theorem}
\begin{theorem}[Chinese remainder theorem]
$N = pq$ where $\gcd(p,q)=1$.
\[\mathbb{Z}_N \simeq \mathbb{Z}_p \times \mathbb{Z}_q\;\;\text{and}\;\;\mathbb{Z}_N^* \simeq \mathbb{Z}_p^* \times \mathbb{Z}_q^* .\]
$f$ maps $x \in \{0,\dotsc,N-1\}$ to pairs $(x_p,x_q):$
\[ f(x) \overset{\text{def}}{=} ([x \bmod p],[x \bmod q]). \]
$f$ is an isomorphism from $\mathbb{Z}_N$ to $\mathbb{Z}_p \times \mathbb{Z}_q$ and 
$\mathbb{Z}_N^*$ to $\mathbb{Z}_p^* \times \mathbb{Z}_q^*$.
\end{theorem}
If $f(x)=(x_p,x_q)$, $x \leftrightarrow (x_p,x_q) = ([x \bmod p], [x \bmod q])$.
\end{frame}
\begin{frame}\frametitle{Using the Chinese Remainder Theorem}
Compute $g=g_1\circ_{\mathbb{G}} g_2$ [$g \equiv g_1 \times g_2 \pmod N$]:
\begin{enumerate}
\item Compute $h_1=f(g_1)$ and $h_2=f(g_2)$;
\item Compute $h=h_1 \circ_{\mathbb{H}} h_2$;
\item Compute $g = f^{-1}(h)$.
\end{enumerate}
\begin{exampleblock}{Compute $14\cdot 13 \bmod 15$}
$[14\cdot 13 \bmod 15] \leftrightarrow (4,2)\cdot (3,1) = ([4\cdot 3 \bmod 5],[2\cdot 1 \bmod 3])$ $=(2,2) \leftrightarrow 2$.
\end{exampleblock}
\end{frame}
\begin{frame}\frametitle{Using the Chinese Remainder Theorem (Cont.)}
Convert $(x_p,x_q)$ to its representation modulo $N$:
\begin{enumerate}
\item Compute $X,Y$ such that $Xp+Yq=1$.
\item $1_p = [Yq \bmod N]$ and $1_q = [Xp \bmod N]$.
\item Compute $x = [(x_p\cdot 1_p+x_q\cdot 1_q) \bmod N]$.
\end{enumerate}
\begin{exampleblock}{Find the representation of $([4 \bmod 5],[3 \bmod 7])$ modulo $35$.}
Use extended Euclidean algorithm, $3\cdot 5-2\cdot 7 =1$.\\
$1_p = [(-2\cdot 7) \bmod 35]=21$ and $1_q = [3\cdot 5 \bmod 35] = 15$.\\
$(4,3) \leftrightarrow [4\cdot 1_p + 3\cdot 1_q \bmod 35] = 24$.
\end{exampleblock}
\begin{exampleblock}{Compute $[29^{100} \bmod 35]$}
$29 \leftrightarrow ([1 \bmod 5],[-1 \bmod 7])$, $[29^{100} \bmod 35] \leftrightarrow (1,-1)^{100} = (1,1) \leftrightarrow 1$.
\end{exampleblock}
\end{frame}
\end{comment}
\begin{frame}\frametitle{Arithmetic algorithms}
\begin{itemize}
\item \textbf{Addition/subtraction}: linear time $O(n)$.
\item \textbf{Mulplication}: naively $O(n^2)$. Karatsuba (1960): $O(n^{\log_2 3})$\\
Basic idea: $(2^bx_1+x_0) \times (2^by_1+ y_0)$ with 3 mults.\\
Best (asymptotic) algorithm: about $O(n\log n)$.
\item \textbf{Division with remainder}: $O(n^2)$.
\item \textbf{Exponentiation}: $O(n^3)$.
\end{itemize}
\begin{algorithm}[H]
\SetKwInOut{Input}{input}
\SetKwInOut{Output}{output}
\SetKw{KwB}{break}
\SetKw{KwH}{halt}
\DontPrintSemicolon
\caption{Exponentiating by Squaring}
\Input{$g \in G$; exponent $x=[x_nx_{n-1}\dots x_2x_1x_0]_2$}
\Output{$g^x$}
\BlankLine
$y \gets g; z \gets 1$\;
\For{$i = 0$ \KwTo $n$}{
  \lIf{$x_i == 1$}{$z \gets z \times y$}\;
  $y \gets y^2$\;
}
\Return $z$
\end{algorithm}
\end{frame}

\section{Primes and Factoring}
\begin{frame}\frametitle{Integer Factorization/Factoring}
%The weak factoring experiment $\mathsf{wFactor}_{\mathcal{A}}(n)$:
%\begin{enumerate}
%\item Choose two $n$-bit integers $x_1, x_2$ at random.
%\item $N := x_1\cdot x_2$.
%\item $\mathcal{A}$ is given $N$, and outputs $x_1',x_2'$.
%\item $\mathsf{wFactor}_{\mathcal{A}}(n) = 1 $ if $x_1'\cdot x_2' = N$ and 0 otherwise.
%\end{enumerate}

\begin{quote}
``The problem of distinguishing prime numbers from composite numbers and of resolving the later into their prime factors is known to be one of the most important and useful in arithmetic.'' -- Gauss (1805)
\end{quote}

The ``hardest'' numbers to factor seem to be those having only large prime factors.
\begin{itemize}
\item The best-known algorithm is the \textbf{general number field sieve} [Pollard] with time $\mathcal{O}(\exp(n^{1/3}\cdot(\log n)^{2/3}))$.
\item RSA Factoring Challenge: RSA-768 (232 digits)
\begin{itemize}
\item Two years on hundreds of machines (2.2GHz/2GB, 1500 years)
\item Factoring a 1024-bit integer: about 1000 times harder.
\end{itemize}
\end{itemize}
\end{frame}
\begin{frame}\frametitle{Generating Random Primes}
\begin{algorithm}[H]
\SetKwInOut{Input}{input}
\SetKwInOut{Output}{output}
\SetKw{KwB}{break}
\SetKw{KwH}{halt}
\DontPrintSemicolon
\caption{Generating a random prime}
\Input{Length $n$; parameter $t$}
\Output{A random $n$-bit prime}
\BlankLine
\For{$i = 1$ \KwTo $t$}{
  $p' \gets \{0,1\}^{n-1}$\;
  $p := 1\| p'$\;
  \lIf{$p$ is prime}{\Return $p$}\;
}
\Return fail
\end{algorithm}
To show its efficiency, we need understand two issues:
\begin{itemize}
\item the probability that a randomly-selected $n$-bit integer is prime.
\item how to efficiently test whether a given integer $p$ is prime.
\end{itemize}
\end{frame}
\begin{frame}\frametitle{The Distribution of Prime}
\begin{theorem}[Prime number theorem]
$\exists$ a constant $c$ such that, $\forall n>1$, a randomly selected $n$-bit number is prime with probability at least $c/n$.
\end{theorem}
The probability that a prime is \emph{not} chosen in $t = n^2/c$ iterations is
\[ \left( 1-\frac{c}{n} \right)^t = \left( \left( 1-\frac{c}{n} \right)^{n/c} \right)^n \le \left( e^{-1} \right)^n = e^{-n}.
\]
The algorithm will fail with a negligible probability.
\end{frame}
\begin{frame}\frametitle{Testing Primality}
\begin{itemize}
\item \textbf{Trial division}: Divide $N$ by $a=2,3,\dotsc,\sqrt{N}.$
\item \textbf{Probabilistic algorithm for approximately computing}:
\begin{itemize}
\item Atlantic City algorithm with two-sided error. 
\item Monte Carlo algorithm with one-sided error.
\item Las Vegas algorithm with zero-sided error.
\end{itemize}
\item \textbf{Fermat primality test}: $a^{N-1} \equiv 1 \pmod N$.
\item $a$ is a \textbf{witness} that $N$ is composite if $a^{N-1} \not \equiv 1 \pmod N$.
\item $a$ is a \textbf{liar} if $N$ is composite and $a^{N-1} \equiv 1 \pmod N$.
\item \textbf{Carmichael numbers}: composite numbers without witnesses.
\end{itemize}
\begin{theorem}
If $\exists$ a witness, then at least half the elements of $\mathbb{Z}_N^*$ are witnesses.
\end{theorem}
\end{frame}
\begin{comment}
\begin{frame}\frametitle{The Miller-Rabin Primality Test}
$N-1=2^ru$, $u$ is odd. $a \in \mathbb{Z}^*_N$ is a \textbf{strong witness} if
\begin{enumerate}
\item $a^u \neq \pm 1$, and
\item $a^{2^iu} \neq -1$ for $i\in\{1,\dotsc,r-1\}$.
\end{enumerate}
\begin{lemma}
$x \in \mathbb{Z}^*$ is a \textbf{square root of 1 modulo} $N$ if $x^2 \equiv 1 \pmod N$. If $N$ is an odd prime then the only $x$ are $[\pm 1 \bmod N]$.
\end{lemma}
\begin{theorem}
$N$ is an odd, composite number that is not a prime power. Then at least half the elements of $\mathbb{Z}^*_N$ are strong witnesses.
\end{theorem}
\begin{theorem}
If $N$ is prime, then the Miller-Rabin test always outputs ``prime''. If $N$ is composite, then the algorithm outputs ``prime'' with probability at most $2^{-t}\;$\footnote{Actually, it is at most $4^{-t}$.}.
\end{theorem}
\end{frame}
\begin{frame}\frametitle{Describing The Algorithm}
\begin{algorithm}[H]
\SetKwInOut{Input}{input}
\SetKwInOut{Output}{output}
\SetKw{KwC}{compute}
\SetKw{KwL}{LOOP}
\DontPrintSemicolon
\caption{The Miller-Rabin primality test}
\Input{Integer $N>2$ and parameter $t$}
\Output{A decision as to wether $N$ is prime or composite}
\BlankLine

\lIf{$N$ is a perfect power}{\Return ``composite''}\;
\KwC $r\ge 1$ and $u$ odd such that $N-1 = 2^ru$\;
\KwL: \For{$s = 1$ \KwTo $t$}{
  $a \gets \{2,\dotsc,N-2\}$\;
  $x = a^u \bmod N$\;
  \lIf{$x = \pm 1$}{do next \KwL}\;
  \For{$i = 1$ \KwTo $r$} {
    $x = x^2 \bmod N$\;
    \lIf{$x = -1 $}{do next \KwL}\;
%    \lIf{$x = 1 $}{\Return ``composite''}\;
  }
  \Return ``composite''\;
}
\Return ``prime''
\end{algorithm}
\end{frame}
\end{comment}
\begin{frame}\frametitle{Examples of Primality Tests}
\begin{exampleblock}{Liars in Fermat primality test}
$2^{340} \equiv 1 \pmod {341}$,  but $341 = 11\cdot 31$.\\  
$5^{560} \equiv 1 \pmod {561}$,  but $561 = 3\cdot 11\cdot 17$.\\
Carmichael numbers $< 10000$: \\  
561,  1105,  1729,  2465,  2821,  6601,  8911.
\end{exampleblock}
%\begin{exampleblock}{Examples of Miller-Rabin test}
%Carmichael number $1729=7\cdot 13\cdot 19$. \\$1729-1 = 1728 = 2^6\cdot 27$. So $r = 6, u = 27$. $a=671$.
%\begin{align*}
%671^{27} &\equiv 1084 \pmod {1729} \\
%671^{27\cdot 2} &\equiv 1065 \pmod {1729}\\
%671^{27\cdot 2^2} &\equiv 1 \pmod {1729}\\
%\end{align*}

%\end{exampleblock}
\end{frame}
\begin{frame}\frametitle{The Factoring Assumption}
Let $\mathsf{GenModulus}(1^n)$ be a polynomial-time algorithm that, on input $1^n$, outputs $(N,p,q)$ where $N=pq$, and $p,q$ are $n$-bit primes except with probability negligible in $n$.
\newline

The factoring experiment $\mathsf{Factor}_{\mathcal{A},\mathsf{GenModulus}}(n)$:
\begin{enumerate}
\item Run $\mathsf{GenModulus}(1^n)$ to obtain $(N,p,q)$.
\item $\mathcal{A}$ is given $N$, and outputs $p', q'>1$.
\item $\mathsf{Factor}_{\mathcal{A},\mathsf{GenModulus}}(n) = 1$ if $p'\cdot q'=N$, and 0 otherwise.
\end{enumerate}
\begin{definition}
\textbf{Factoring is hard relative to} $\mathsf{GenModulus}$ if $\forall$ \textsc{ppt} algorithms $\mathcal{A}$, $\exists$ $\mathsf{negl}$ such that
\[ \Pr[\mathsf{Factor}_{\mathcal{A},\mathsf{GenModulus}}(n) = 1] \le \mathsf{negl}(n).\] 
\end{definition}
\end{frame}
\begin{frame}\frametitle{Algorithms for Factoring}
\begin{itemize}
\item \textbf{Factoring} $N=pq$. $p,q$ are of the same length $n$.
\item \textbf{Trial division}: $\mathcal{O}(\sqrt{N}\cdot \mathsf{polylog}(N))$.
\item \textbf{Pollard's $p-1$} method: effective when $p-1$ has ``small'' prime factors.
\item \textbf{Pollard's rho} method: $\mathcal{O}(N^{1/4}\cdot \mathsf{polylog}(N))$.
\item \textbf{Quadratic sieve} algorithm [Carl Pomerance]: sub-exponential time $\mathcal{O}(\exp(\sqrt{n\cdot \log n}))$.
\item The best-known algorithm is the \textbf{general number field sieve} [Pollard] with time $\mathcal{O}(\exp(n^{1/3}\cdot(\log n)^{2/3}))$.
\end{itemize}
\end{frame}
\begin{comment}
\begin{frame}\frametitle{Pollard's $p-1$ Method}
\textbf{Idea}: Fermat's little theorem: $y = x^{(p-1)\cdot k} \equiv 1 \pmod p$. Then $(y-1) \equiv 0 \pmod p$ and $p \mid (y-1)$. So $p = \gcd(y-1,N)$. To make the exponent a large multiple of $(p-1)$:
\[ M = lcm(\{ i | i \le B \}) = \prod_{\text{prime}\;i \le B}i^{\lfloor \log_iB \rfloor}.\]
If $p-1$ has only ``small'' factors, then the bound $B$ will be small.
\begin{algorithm}[H]
\SetKwInOut{Input}{input}
\SetKwInOut{Output}{output}
\SetKw{KwC}{compute}
\SetKw{KwL}{LOOP}
\DontPrintSemicolon
\caption{Pollard's $p-1$ algorithm for factoring}
\Input{Integer $N$}
\Output{A non-trivial factor of $N$}
\BlankLine

$x \gets \mathbb{Z}^*_N$\;
$y := [x^M \bmod N]$\;
$p := \gcd(y-1,N)$\;
\lIf{$p \notin \{1,N\}$}{\Return $p$}\;
\end{algorithm}
\end{frame}
\begin{frame}\frametitle{Pollard's Rho ($\rho$) Method}
\textbf{Idea}: Using the improved birthday attack\footnote{Floyd's cycle-finding algorithm (the ``tortoise and the hare'' algorithm).} to find $x,x'$ such that $x \neq x' \land x \equiv x' \pmod p$. Then $p \mid (x-x')$, $p = \gcd(x-x',N)$.
$F(x) = x^2+b$, where $b \not \equiv 0,-2 \pmod N$.
\begin{algorithm}[H]
\SetKwInOut{Input}{input}
\SetKwInOut{Output}{output}
\SetKw{KwC}{compute}
\SetKw{KwL}{LOOP}
\DontPrintSemicolon
\caption{Pollard's rho algorithm for factoring}
\Input{Integer $N$}
\Output{A non-trivial factor of $N$}
\BlankLine

$x_0 \gets \mathbb{Z}^*_N$\;
\For{$i=1$ \KwTo $2^{n/2}$}{
$x_i := [F(x_{i-1}) \bmod N]$\;
$x_{2i} := [F(F(x_{2i-2})) \bmod N]$\;
$p := \gcd(x_{2i}-x_i,N)$\;
\lIf{$p \notin \{1,N\}$}{\Return $p$}\;
}
\end{algorithm}
\end{frame}
\begin{frame}\frametitle{Proof of Pollard's $\rho$ Method}
\begin{lemma}
Let $x_1,\dotsc$ be a sequence with $x_m \equiv F(x_{m-1}) \pmod N$. $F$ satisfies that $x \equiv x' \pmod N \implies F(x) \equiv F(x') \pmod N$. If $x_I \equiv x_J\pmod p$ with $I < J$, then $\exists$ $i < J$ such that $x_{i} \equiv x_{2i} \pmod p$.
\end{lemma}
\begin{figure}
\begin{center}
\begin{tikzpicture}
\draw[-] (0,0) -- (9.5,0);
\foreach \x in {0,1.5,...,7.6}{
\fill[red!60] (\x,-3pt) rectangle +(1.43,-3pt);
}
\node (I) at (3.6,0.2) [above] {$I$};
\node (J) at (5.1,0.2) [above] {$J$};
\foreach \x in {3,4.5,...,7.5}{
\fill[blue!60] (\x+0.6,3pt) rectangle +(1.43,3pt);
}
\node (i) at (4.5,-0.2) [below] {$i$};
\node (2i) at (9,-0.2) [below] {$2i$};
\node (x0) at (-0,0) [left] {$x_0$};
\end{tikzpicture}
\end{center}
\end{figure}
\begin{proof}
See the proof of improved birthday attack.
\end{proof}
According to the lemma of birthday problem, given a sequence of length $O(N^{1/4})$, find such pair with probability $1/4$.
\end{frame}
\begin{frame}\frametitle{Example of Pollard's $p-1$ and $\rho$ methods}
\begin{exampleblock}{Factorizing $N=5917$ with Pollard's $p-1$ method}
Choose $B=5$, $M=lcm(1,2,3,4,5)=60$.\\
For $x=2$, $y \equiv x^M \equiv 2^{60} \equiv 3417 \pmod{5917}$.\\
$p = gcd(y-1,N) = \gcd(3416,5917) = 61$.
\end{exampleblock}
\begin{exampleblock}{Factorizing $N=8051$ with Pollard's $\rho$ method}
$f(x) = x^2+1$, $x_0=2$.\\
\begin{center}
\begin{tabular}{|c|c|c|c|} \hline
$i$  & $x_i$ & $x_{2i}$ & $\gcd(x_{2i}-x_i,N)$ \\ \hline
1 & 5 & 26 & 1 \\
2 & 26 & 7474 & 1 \\
3 & 677 & 871 & 97 \\ \hline
\end{tabular}	
\end{center}
\end{exampleblock}
\end{frame}
\begin{frame}\frametitle{The Quadratic Sieve Algorithm}
\textbf{Idea}: Find $x,y$ with $x^2 \equiv y^2 \pmod N$ and $x \not \equiv \pm y \pmod N$. $x^2-y^2 \equiv 0 \pmod N\implies (x+y)(x-y) \equiv 0 \pmod N$.\\
$\gcd(x+y,N)$ and $\gcd(x-y,N)$ will give $p$.\\
\textbf{Finding congruence of squares}: \\
%Find $x_i^2 \equiv y_i \pmod N$ for $i=1,2,\dotsc,r$ and $y_1y_2\cdots y_r = c^2$. \\
%$(x_1x_2\cdots x_r)^2 \equiv y_1y_2\cdots y_r = c^2 \pmod N$.\\
\begin{enumerate}
\item Choose a factor base $B = \{p_1,\dotsc,p_k\}$ of prime numbers. 
\item Use `\textbf{sieve theory}' to find $\ell = k+1$ distinct $x_1,\dotsc,x_\ell$ for which $[x_i^2 \bmod N]$ decompose into the elements of $B$: $x_i^2 \equiv \prod^k_{j=1} p_j^{e_j} \pmod N$.
\item Write $x_i^2$ as an exponent vector $\langle e_{i,1},\dotsc,e_{i,k}\rangle \pmod 2$.
\item Find the addition of vectors = the zero vector $\pmod 2$.\\
$X=\{x_{\ell_1},\dotsc,x_{\ell_n}\}$. $\forall i$, $E_i = \sum_{j=1}^ne_{\ell_j,i} \equiv 0 \pmod 2$.
\item Find a pair: $x = \prod_{i=1}^nx_{\ell_i} \not \equiv y=\prod_{i=1}^kp_i^{E_i/2} \pmod N$.  
\end{enumerate}
\end{frame}
\begin{frame}\frametitle{Example of Quadratic Sieve Algorithm}
\begin{exampleblock}{Factorizing $N=377753$ with quadratic sieve algorithm}
$B = \{2,13,17,23,29\}$.
\begin{align*}
620^2 &\equiv 17^2\cdot 23 \pmod N\\
621^2 &\equiv 2^4\cdot 17\cdot 29 \pmod N\\
645^2 &\equiv 2^7\cdot 13\cdot 23 \pmod N\\
655^2 &\equiv 2^3\cdot 13\cdot 17\cdot 29 \pmod N
\end{align*}
\[ [620\cdot 621\cdot 645\cdot 655 \bmod N]^2 \equiv [2^7\cdot 13\cdot 17^2\cdot 23\cdot 29\bmod N]^2 \]
\[ \implies 127194^2 \equiv 45335^2 \pmod N,\]
Computing $\gcd(127194-45335,377753)=751$.
\end{exampleblock}
\end{frame}
\end{comment}
\section{RSA Assumption}
\begin{frame}\frametitle{The RSA Problem}
\begin{block}{Recall group exponentiation on $\mathbb{Z}^*_N$}
Define function $f_e\;:$ $\mathbb{Z}^*_N \to \mathbb{Z}^*_N$ by $f_e(x) =[x^e \bmod N]$. \\ If $\gcd(e,\phi(N))=1$, then $f_e$ is a permutation. \\
If $d = [e^{-1} \bmod \phi(N)]$, then $f_d$ is the inverse of $f_e$.\\
\textbf{$e$'th root of $c$}: $g^e = c$, $g = c^{1/e} = c^{d}$. 
\end{block}
\textbf{Idea}: factoring is hard\\ $\implies$ for $N=pq$, finding $p,q$ is hard\\ $\implies$ computing $\phi(N)=(p-1)(q-1)$ is hard\\ $\implies$ computations modulo $\phi(N)$ is not available\\ 
\alert{\textbf{There is a gap.}}\\
$\implies$ \textbf{RSA problem} [Rivest, Shamir, and Adleman] is hard:\\
Given $y \in \mathbb{Z}^*_N$, compute $y^{-e}$, $e^{\text{th}}$-root of $y$ modulo $N$.
\begin{alertblock}{Open problem}
RSA problem is easier than factoring?
\end{alertblock}
\end{frame}
\begin{frame}\frametitle{Generating RSA Problem}
\begin{algorithm}[H]
\SetKwInOut{Input}{input}
\SetKwInOut{Output}{output}
\SetKw{KwF}{find}
\SetKw{KwC}{compute}
\DontPrintSemicolon
\caption{$\mathsf{GenRSA}$}
\Input{Security parameter $1^n$}
\Output{$N,e,d$}
\BlankLine
$(N,p,q) \gets \mathsf{GenModulus}(1^n)$\;
$\phi(N) := (p-1)(q-1)$\;
\KwF $e$ such that $\gcd(e,\phi(N))=1$\;
\KwC $d := [e^{-1} \bmod \phi(N)]$\;
\Return $N,e,d$\;
\end{algorithm}
\end{frame}
\begin{frame}\frametitle{The RSA Assumption}
The RSA experiment $\mathsf{RSAinv}_{\mathcal{A},\mathsf{GenRSA}}(n)$:
\begin{enumerate}
\item Run $\mathsf{GenRSA}(1^n)$ to obtain $(N,e,d)$.
\item Choose $y \gets \mathbb{Z}^*_N$.
\item $\mathcal{A}$ is given $N,e,y$, and outputs $x \in \mathbb{Z}^*_N$.
\item $\mathsf{RSAinv}_{\mathcal{A},\mathsf{GenRSA}}(n)=1$ if $x^e \equiv y \pmod N$, and 0 otherwise.
\end{enumerate}
\begin{definition}
\textbf{RSA problem is hard relative to} $\mathsf{GenRSA}$ if $\forall$ \textsc{ppt} algorithms $\mathcal{A}$, $\exists$ $\mathsf{negl}$ such that
\[ \Pr[\mathsf{RSAinv}_{\mathcal{A},\mathsf{GenRSA}}(n) = 1] \le \mathsf{negl}(n).\]
\end{definition}
\end{frame}

\begin{comment}
\begin{frame}\frametitle{Constructing One-Way Functions}
\begin{algorithm}[H]
\SetKwInOut{Input}{input}
\SetKwInOut{Output}{output}
\SetKw{KwC}{compute}
\SetKw{KwL}{LOOP}
\DontPrintSemicolon
\caption{Algorithm computing $f_{\mathsf{GenModulus}}$}
\Input{String $x$}
\Output{String $N$}
\BlankLine

\KwC $n$ such that $p(n) \le \abs{x} < p(n+1)$\;
\KwC $(N,p,q) := \mathsf{GenModulus}(1^n;x)$\;
\tcc{run $\mathsf{GenModulus}(1^n)$ using $x$ as the random tape}
\Return $N$\;
\end{algorithm}
Reduce the factoring problem to the inverting problem.
\begin{theorem}
If the factoring problem is hard relative to $\mathsf{GenModulus}$, then $f_{\mathsf{GenModulus}}$ is a one-way function.
\end{theorem}
\end{frame}
\begin{frame}\frametitle{Constructing One-Way Permutations}
\begin{construction}
Define a family of permutations with $\mathsf{GenRSA}$:
\begin{itemize}
\item $\mathsf{Gen}$: on input $1^n$, run $\mathsf{GenRSA}(1^n)$ to obtain $(N,e,d)$ and output $I=\langle N,e \rangle$, Set $\mathcal{D}_I = \mathbb{Z}^*_N$.
\item $\mathsf{Samp}$: on input $I=\langle N,e \rangle$, choose a random elements of $\mathbb{Z}^*_N$.
\item $f$: on input $I=\langle N,e\rangle$ and $x \in \mathbb{Z}^*_N$, output $[ x^e \bmod N]$.
\end{itemize}
\end{construction}
Reduce the RSA problem to the inverting problem.
\end{frame}
\end{comment}
\begin{frame}\frametitle{Summary}
\begin{itemize}
\item Primes, modular arithmetic.
%\item Miller-Rabin primality testing.
%\item Factoring, Pollard's $p-1$ and $\rho$ methods.
\item $e^{\mathsf{th}}$-root modulo $N$, RSA.
\end{itemize}
\begin{block}{Textbook}
``\emph{A Computational Introduction to Number Theory and Algebra}''
(Version 2) by Victor Shoup
\end{block}
\end{frame}
\end{document}