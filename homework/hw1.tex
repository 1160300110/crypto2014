% AMS dept HW template example
% v0.04 by Eric J. Malm, 10 Mar 2005
\documentclass[12pt,letterpaper,boxed]{amspset}

% set 1-inch margins in the document
\usepackage[margin=1in]{geometry}

% include this if you want to import graphics files with /includegraphics
\usepackage{graphicx}

% info for header block in upper right hand corner
\name{{\large Name: }\underline{ \hspace{3cm} }}
\class{{\large ID: }\underline{ \hspace{3cm} }}
\assignment{{\large Grade:}\underline{ \hspace{3cm} }}
\duedate{03/22/2011}

\begin{document}

\problemlist{Classic cipher, Perfectly-Secret Encryption\\Private-Key Encryption, Pseudorandomness}


\begin{problem}[1.4]
Show that the shift, Mono-Alphabetic sub., and Vigen\`{e}re ciphers are all trivial to break using a known-plaintext attack. How much known plaintext (how many characters) is needed to completely recover the key for each of the ciphers?  (show how to break the cipher)
\end{problem}

\begin{solution}
Shift:\\
\\
\\
\\
Mono-Alphabetic sub.:\\
\\
\\
\\
Vigen\`{e}re: \\
\\
\\
\end{solution}

\begin{problem}[1.5]
Show that the shift, Mono-Alphabetic sub., and Vigen\`{e}re ciphers are all trivial to break using a chosen-plaintext attack. How much plaintext (how many characters) must be encrypted to completely recover the key? (show your chosen plaintext) 
\end{problem}

\begin{solution}
Shift:\\
\\
\\
\\
Mono-Alphabetic sub.:\\
\\
\\
\\
Vigen\`{e}re: \\
\\
\\
\end{solution}

\begin{problem}[1.6]
What is the index of coincidence of your name in Pinyin (without blank space and ignoring case)?
\end{problem}

\begin{solution}
Name:\\
Letters and their corresponding probabilities in your name:\\
\\
\\
IC = 
\end{solution}

\begin{problem}[2.1]
Prove or refute: For every encryption scheme that is perfectly secret it holds that for every distribution over the message space $\mathcal{M}$, every $m, m' \in \mathcal{M}$, and every $c \in \mathcal{C}$:
\[ \Pr[M=m | C=c] = \Pr[M=m'|C=c].
\]
\end{problem}

\begin{solution}
\vspace{5cm}

\end{solution}

\begin{problem}[2.2]
Study conditions under which the shift, mono-alphabetic sub., and Vigen\`{e}re cipher ciphers are perfectly secret:
\begin{itemize}
\item (a) Prove that if only a single character is encrypted, then the shift cipher is perfectly secret.
\item (b) What is the largest plaintext space $M$ you can find for which the mono-alphabetic sub. cipher provides perfect secrecy?
\item (c) Show how to use the Vigen\`{e}re cipher to encrypt any word of length $t$ so that perfect secrecy is obtained (note: you can choose the length of the key). Prove your answer.
\end{itemize}
\end{problem}

\begin{solution}
(a) Shift:\\
\\
\\
\\
\\
\\
(b) Mono-alphabetic sub.:\\
\\
\\
\\
\\
\\
(c) Vigen\`{e}re cipher.:\\
\\
\\
\\
\\
\\
\\
\\
\end{solution}

\begin{problem}[3.1]
The best algorithm known today for finding the prime factors of an $n$-bit number runs in time $2^c\cdot n ^ {\frac{1}{3}(\log n)^{\frac{1}{3}}}$. Assuming 4Ghz computers and $c=1$, estimate the size of numbers that cannot be factored for the next 100 years.
\end{problem}

\begin{solution}
(Do not only give the value of $n$, show the process of solving it.)\\
\\
\\
\\
\\
\\
\end{solution}

\begin{problem}[3.2]
Prove that Definition 1 (see handout `3privatekey.pdf') cannot be satisfied if $\Pi$ can encrypt arbitrary-length messages and the adversary is not restricted to output equal-length messages in experiment $\mathsf{PrivK}^{\mathsf{eav}}_{\mathcal{A},\Pi}(n)$.
\end{problem}

\begin{solution}
(Show what the adversary would output, and the probability the experiment will success.)
\\
\\
\\
\\
\\
\\
\\
\end{solution}

\begin{problem}[3.3]
Assuming the existence of a pseudorandom function, prove that there exists an encryption scheme that has indistinguishable multiple encryptions in the presence of an eavesdropper (i.e. Definition 8), but is not CPA-secure (i.e. Definition 10). (see handout `3privatekey.pdf')
\end{problem}

\begin{solution}
{\small Hint: You will need to use the fact that in a CPA the adversary can choose its queries to the encryption oracle adaptively (i.e., new query may be constructed from previous queries).}\\
\\
\\
\\
\\
\\
\\
\\
\\
\end{solution}

\begin{problem}[3.4]
Present a construction of a variable output-length pseudorandom generator from any pseudorandom function. Prove that your construction satisfies Definition 7 (see handout `3privatekey.pdf').
\end{problem}

\begin{solution}
\vspace{5cm}
\end{solution}

\begin{problem}[3.5]
Present formulas for decryption of all the different modes of operation for encryption. For which modes can decryption be parallelized?
\end{problem}

\begin{solution}
ECB:\\
\\
\\
\\
CBC:\\
\\
\\
\\
OFB:\\
\\
\\
\\
CRT:\\
\\
\\
\\
\end{solution}

\begin{problem}[3.6]
Show that the CBC, OFB and CRT modes do not yield CCA-secure encryption schemes (regardless of F).
\end{problem}

\begin{solution}
CBC:\\
\\
\\
\\
\\
OFB:\\
\\
\\
\\
\\
CRT:\\
\\
\\
\\
\end{solution}


\end{document}
