\documentclass[11pt]{article}

% set 1-inch margins in the document
\usepackage[margin=1in]{geometry}
\usepackage{amsthm}
\usepackage{amsmath}
\usepackage{amssymb,amsfonts}
\theoremstyle{definition}

% include this if you want to import graphics files with /includegraphics
\usepackage{graphicx}
\providecommand{\abs}[1]{\lvert#1\rvert}

% info for header block in upper right hand corner

\newtheorem{problem}{Problem}

\title{HIT --- Cryptography --- Homework 5}

\begin{document}

\maketitle

%\begin{problem}
%This question concerns the Euler phi function.
%\begin{enumerate}
%\item Let  $p$ be a prime and $e \ge 1$ an integer. Show that $\phi(p^e) = p^{e-1}(p-1)$.
%\item Let $p,q$ be relatively prime. Show that $\phi(pq) = \phi(p)\cdot \phi(q)$. (You may use the Chinese remainder theorem.)
%\item Prove Theorem: $N = \prod_ip_i^{e_i}$, $\{p_i\}$ are distinct primes, $\phi(N) = \prod_ip_i^{e_i-1}(p_i-1)$.
%\end{enumerate}
%\end{problem}

\begin{problem}
Compute $[101^{4,800,000,023} \bmod 35]$ (by hand).
\end{problem}

\begin{problem}
Let $N=pq$ be a product of two distinct primes. Show that if $\phi(N)$ and $N$ are known, then it is possible to compute $p$ and $q$ in polynomial time.
\end{problem}

\begin{problem}
For an RSA public key  $\langle N, e \rangle$, we have an algorithm $\mathcal{A}$ that always correctly computes $LSB(x)$ given $[ x^e \mod N]$. Design an algorithm that computes $x$ from $[ x^e \mod N]$.
\end{problem}

\begin{problem}
Consider the following key-exchange protocol:
\begin{enumerate}
\item Alice chooses $k,r \gets \{0,1\}^n$ at random, and sends $s:=k\oplus r$ to Bob.
\item Bob chooses $t \gets \{0,1\}^n$ at random and sends $u := s\oplus t$ to Alice.
\item Alice computes $w := u\oplus r$ and sends $w$ to Bob.
\item Alice outputs $k$ and Bob computes $w \oplus t$.
\end{enumerate}
Show that Alice and Bob output the same key. Analyze the security of the scheme (i.e. either prove its security or show a concrete attack by an eavesdropper).
\end{problem}

\begin{problem}
Consider the following public-key encryption scheme. The public key is $(\mathbb{G},q,g,h)$ and the private key is $x$, generated exactly as in the El Gamal encryption scheme. In order to encrypt a bit $b$, the sender does the following:
\begin{itemize}
\item If $b=0$ then choose a random $y \gets \mathbb{Z}_q$ and compute $c_1= g^y$ and $c_2=h^y$. The ciphertext is $\langle c_1,c_2\rangle$.
\item If $b=1$ then choose independent random $y,z \gets \mathbb{Z}_q$ and compute $c_1= g^y$ and $c_2=g^z$, and set the ciphertext is $\langle c_1,c_2\rangle$.
\end{itemize}
(a) Show that it is possible to decrypt efficiently given knowledge of $x$. (b) Prove that this encryption scheme is CPA-secure if the decisional Diffie-Hellman problem is hard relative to $\mathcal{G}$
\end{problem}

\begin{problem}
The natural way of applying hybrid encryption to the El Gamal encryption scheme is as follows. The public key is $pk = \langle \mathbb{G},q,g,h\rangle $ as in the El Gamal scheme, and to encrypt a message $m$ the sender chooses random $k \gets \{0,1\}^n$ and sends
\[ \langle g^r, h^r\cdot k, \mathsf{Enc}_k(m)\rangle, \]
where $r\gets \mathbb{Z}_q$ is chosen at random and $\mathsf{Enc}$ represents a private-key encryption scheme. Suggest an improvement that results in a shorter ciphertext containing only a \emph{single} group element followed by a private-key encryption of $m$.
\end{problem}

\begin{problem}
For each of the following variants of the definition of security for signatures, state whether textbook RSA is secure and prove your answer: 
\begin{itemize}
\item (a) In this first variant, the experiment is as follows: the adversary is given the public key $pk$ and a random message $m$. The adversary is then allowed to query the signing oracle once on a single message that does not equal $m$. Following this, the adversary outputs a signature $\sigma$ and succeeds if $\mathsf{Vrfy}_{pk}(m,\sigma)=1$. As usual, security is said to hold if the adversary can succeed in this experiment with at most negligible probability.	
\item (b) The second variant is as above, except that the adversary is not allowed to query the signing oracle at all.
\end{itemize}
\end{problem}

\begin{problem}
Consider the Lamport one-time signature scheme. Describe an adversary who obtains signatures on two messages of its choice and can then forge signatures on any message it likes.
\end{problem}

\end{document}