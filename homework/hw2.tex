\documentclass[11pt]{article}

% set 1-inch margins in the document
\usepackage[margin=1in]{geometry}
\usepackage{amsthm}
\theoremstyle{definition}

% include this if you want to import graphics files with /includegraphics
\usepackage{graphicx}

% info for header block in upper right hand corner

\newtheorem{problem}{Problem}

\title{HIT --- Cryptography --- Homework 2}

\begin{document}

\maketitle

\begin{problem}
Prove that the indistinguishable encryptions in the presence of an eavesdropper cannot be satisfied if $\Pi$ can encrypt arbitrary-length messages and the adversary is not restricted to output equal-length messages in experiment $\mathsf{PrivK}^{\mathsf{eav}}_{\mathcal{A},\Pi}(n)$.
\end{problem}

\begin{problem}
Assuming the existence of a pseudorandom function, prove that there exists an encryption scheme that has indistinguishable multiple encryptions in the presence of an eavesdropper, but is not CPA-secure.
{\small Hint: You will need to use the fact that in a CPA the adversary can choose its queries to the encryption oracle adaptively (i.e., new query may be constructed from previous queries).}
\end{problem}

\begin{problem}
Present a construction of a variable output-length pseudorandom generator from any pseudorandom function. Prove that your construction is PRG.
\end{problem}

\begin{problem}
Present formulas for decryption of all the different modes of operation for encryption: ECB, CBC, OFB, CRT. For which modes can decryption be parallelized?
\end{problem}

\begin{problem}
Show that the CBC, OFB and CRT modes do not yield CCA-secure encryption schemes (regardless of F).
\end{problem}

\end{document}
