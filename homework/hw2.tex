% AMS dept HW template example
% v0.04 by Eric J. Malm, 10 Mar 2005
\documentclass[12pt,letterpaper,boxed]{amspset}

% set 1-inch margins in the document
\usepackage[margin=1in]{geometry}

% include this if you want to import graphics files with /includegraphics
\usepackage{graphicx}

% info for header block in upper right hand corner
\name{{\large Name: }\underline{ \hspace{3cm} }}
\class{{\large ID: }\underline{ \hspace{3cm} }}
\assignment{{\large Grade:}\underline{ \hspace{3cm} }}
\duedate{04/06/2011}

\begin{document}

\problemlist{Message Authentication Codes, Collision-Resistant Hash Functions,\\Block Ciphers, One-Way Function}

\begin{problem}[4.1]
Let $F$ be a pseudorandom function. Show that the following MAC for messages of length $2n$ is insecure: The shared key is a random $k\in \{0,1\}^n$. To authenticate a message $m_1\| m_2$ with $\abs{m_1}=\abs{m_2}=n$, compute the tag $\langle F_k(m_1), F_k(F_k(m_2))\rangle$. 
\end{problem}

\begin{solution}
\vspace{3cm}
\end{solution}

\begin{problem}[4.2]
Show that the basic CBC-MAC construction is not secure when used to authenticate messages of different lengths.
\end{problem}

\begin{solution}
\vspace{3cm}
\end{solution}

%\begin{problem}[4.3]
%Provide formal definitions for second pre-image resistance and pre-image resistance. Formally prove that any hash function that is collision resistant is second pre-image resistant, and that any hash function that is second pre-image resistant is pre-image resistant.
%\end{problem}

%\begin{solution}
%Definition of the second pre-image resistance:

%\vspace{5cm}
%\noindent Proof of its security based on collision resistant:\\

%\vspace{3cm}
%\noindent Definition of the pre-image resistance:\\

%\vspace{3cm}
%\noindent Proof of its security based on second pre-image resistant:\\

%\vspace{3cm}
%\end{solution}

\begin{problem}[4.4]
Let $(\mathsf{Gen},H)$ be a collision-resistant hash function. Is $(\mathsf{Gen},\hat{H})$ defined by $(\hat{H}^s(x) \overset{\text{def}}{=} H^s(H^s(x))$ necessarily collision resistant? Prove your answer. 
\end{problem}

\begin{solution}
Proof:\\

\vspace{3cm}
\end{solution}

\begin{problem}[4.5]
For each of following modifications to the Merkle-Damg\r{a}rd transform, determine whether the result is collision resistant or not. If yes, provide a proof; if not, demonstrate an attack.
\end{problem}

\begin{solution}
(a) Modify the construction so that the input length is not included at all (i.e, output $z_B$ and not $z_{B+1} = h^s(z_B\| L)$).\\

\vspace{4cm}
\noindent (b) Modify the construction so that instead of outputting $z = h^s(z_B\| L)$, the algorithm outputs $z_B\|L$\\

\vspace{4cm}
\noindent (c) Instead of using an $IV$, just start the computation from $x_1$. That is, define $z_1 := x_1$ and then compute $z_i := h^s(z_{i-1}\|x_i)$ for $i=2,\dotsc,B+1$ and output $z_{B+1}$ as before.\\

\vspace{4cm}
\noindent (d) Instead of using a fixed $IV$, set $z_0 := L$ and then compute $z_i := h^s(z_{i-1}\|x_i)$ for $i=1,\dotsc,B$ and output $z_B$.\\

\vspace{4cm}
\end{solution}

\begin{problem}[5.1]
In our attack on a two-round substitution-permutation network, we considered a block length of 64 bits and a network with 16 $S$-boxes that each take a 4-bit input. 
\end{problem}

\begin{solution}
(a) Repeat the analysis for the case of 8 $S$-boxes, each taking an 8-bit input. What is the complexity of the attack now?\\

\vspace{4cm}
\noindent (b) Repeat the analysis again with a 128-bit block length and 16 $S$-boxes that each take an 8-bit input.\\

\vspace{4cm}
\noindent (c) Does the block length make any difference?\\

\vspace{1cm}
\end{solution}

\begin{problem}[5.2]
What is the output of an $r$-round Feistel network when the input is $(L_0, R_0)$ in each of the following two cases: (Show your analysis.)
\end{problem}

\begin{solution}
(a) Each round function $F$ outputs all $0$s, regardless of the input.

\vspace{7cm}
\noindent (b) Each round function $F$ is the identity function:

\vspace{5cm}
\end{solution}

\begin{problem}[5.3]
Show that DES has the property that $DES_k(x) = \overline{DES_{\overline{k}}(\overline{x})}$ for every key $k$ and input $x$ (where $\overline{z}$ denotes the bitwise complement of $z$). This is called the complementarity property of $DES$.
\end{problem}

\begin{solution}
\vspace{4cm}
\end{solution}

%\begin{problem}[5.4]
%Show an improvement to the attack on three-round DES that recovers the key using two input/output pairs but runs in time $2\cdot 2^{28}+2\cdot 2^{12}$.
%\end{problem}

%\begin{solution}
%\vspace{6cm}
%\end{solution}

\begin{problem}[6.1]
Prove that if $f$ is a one-way function, then $g(x_1,x_2) = (f(x_1),x_2)$ where $\abs{x_1} = \abs{x_2}$ is also a one-way function. Observe that $g$ fully reveals half of its input bits, but is nevertheless still one-way.
\end{problem}

\begin{solution}


\vspace{5cm}
\end{solution}

%\begin{problem}[6.2]
%Let $f$ be a one-way function. Is $g(x) = f(f(x))$ necessarily a one-way function? What about $g(x) = (f(x),f(f(x)))$? Prove your answers.
%\end{problem}

%\begin{solution}
%\vspace{5cm}
%\end{solution}
%\begin{problem}[6.3]
%Let $G$ be a pseudorandom generator with expansion factor $\ell(n)=n+1$. Prove that $G$ is a one-way function.
%\end{problem}

%\begin{solution}
%\vspace{5cm}
%\end{solution}

\end{document}