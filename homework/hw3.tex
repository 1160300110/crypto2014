% AMS dept HW template example
% v0.04 by Eric J. Malm, 10 Mar 2005
\documentclass[12pt,letterpaper,boxed]{amspset}

% set 1-inch margins in the document
\usepackage[margin=1in]{geometry}

% include this if you want to import graphics files with /includegraphics
\usepackage{graphicx}

% info for header block in upper right hand corner
\name{{\large Name: }\underline{ \hspace{3cm} }}
\class{{\large ID: }\underline{ \hspace{3cm} }}
\assignment{{\large Grade:}\underline{ \hspace{3cm} }}
\duedate{05/12/2011}

\begin{document}

\problemlist{Factoring, RSA, Discrete Log, Diffie-Hellman, Key Management,\\ 
 Public Key, El Gamal, TDP, Digital Signatures, ROM}

\begin{problem}[7.1]
Let $\mathbb{G}$ be an abelian group. Prove that there is a \emph{unique} identity in $\mathbb{G}$, and that every element $g \in \mathbb{G}$ has a \emph{unique} inverse.
\end{problem}
\begin{solution}
\vspace{3cm}
\end{solution}

\begin{problem}[7.2]
This question concerns the Euler phi function.
\end{problem}

\begin{solution}
(a) Let  $p$ be a prime and $e \ge 1$ an integer. Show that $\phi(p^e) = p^{e-1}(p-1)$.
\vspace{3cm}

\noindent (b) Let $p,q$ be relatively prime. Show that $\phi(pq) = \phi(p)\cdot \phi(q)$. (You may use the Chinese remainder theorem.)
\vspace{4cm}

\noindent (c) Prove Theorem: $N = \prod_ip_i^{e_i}$, $\{p_i\}$ are distinct primes, $\phi(N) = \prod_ip_i^{e_i-1}(p_i-1)$.
\vspace{3cm}

\end{solution}

\begin{problem}[7.3]
Solve the following system of congruences (find $x$ by hand):
\[ 13x \equiv 4 \pmod{99},\;\;\;\;\; 15x \equiv 56 \pmod{101}\]
\end{problem}

\begin{solution}
\vspace{3cm}

\end{solution}



\begin{problem}[7.4]
Compute $[101^{4,800,000,023} \bmod 35]$ (by hand).
\end{problem}

\begin{solution}
\vspace{4cm}

\end{solution}

\begin{problem}[7.5]
Prove that if $\mathbb{G}, \mathbb{H}$ are groups, then $\mathbb{G}\times \mathbb{H}$ is a group.
\end{problem}

\begin{solution}
\vspace{5cm}

\end{solution}

\begin{problem}[7.6]
Let $N=pq$ be a product of two distinct primes. Show that if $\phi(N)$ and $N$ are known, then it is possible to compute $p$ and $q$ in polynomial time.
\end{problem}

\begin{solution}
\vspace{3cm}

\end{solution}

\begin{problem}[7.7]
Prove formally that the hardness of the CDH problem relative to $\mathcal{G}$ implies the hardness of the discrete logarithm problem relative to $\mathcal{G}$.
\end{problem}

\begin{solution}
\vspace{8cm}

\end{solution}

\begin{problem}[9.1]
Consider the following key-exchange protocol:
\begin{enumerate}
\item Alice chooses $k,r \gets \{0,1\}^n$ at random, and sends $s:=k\oplus r$ to Bob.
\item Bob chooses $t \gets \{0,1\}^n$ at random and sends $u := s\oplus t$ to Alice.
\item Alice computes $w := u\oplus r$ and sends $w$ to Bob.
\item Alice outputs $k$ and Bob computes $w \oplus t$.
\end{enumerate}
Show that Alice and Bob output the same key. Analyze the security of the scheme (i.e. either prove its security or show a concrete attack by an eavesdropper).
\end{problem}

\begin{solution}
\vspace{6cm}

\end{solution}

\begin{problem}[10.1]
Assume a public-key encryption scheme for single-bit messages. Show that, given $pk$ and a ciphertext $c$ computed via $c \gets \mathsf{Enc}_{pk}(m)$, it is possible for an unbounded adversary to determine $m$ with probability 1. This shows that perfectly-secret public-key encryption is impossible.
\end{problem}

\begin{solution}
\vspace{3cm}

\end{solution}

\begin{problem}[10.2]
Say a deterministic public-key encryption scheme is used to encrypt a message $m$ that is known to lie in a small set of $\mathcal{L}$ possible values. Show how it is possible to determine $m$ in time linear in $\mathcal{L}$ (assume that encryption of an element takes a single unit of time).
\end{problem}

\begin{solution}
\vspace{3cm}

\end{solution}

\begin{problem}[10.3]
In multiple message eavesdropping experiment in public-key encryption, when $t=2$ (two messages in each vector of messages), prove that there exists a negligible function $\mathsf{negl}$ such that $\frac{1}{2}+\mathsf{negl}(n) \ge \frac{1}{2}\cdot \Pr[\mathcal{A}(c_0^1,c_1^2)=0] + \frac{1}{2}\cdot \Pr[\mathcal{A}(c_1^1,c_1^2)=1].$ (see Page 8 in 8.2pubkey-RSA.pdf)
\end{problem}

\begin{solution}
\vspace{6cm}

\end{solution}

\begin{problem}[10.4]
Consider the following public-key encryption scheme. The public key is $(\mathbb{G},q,g,h)$ and the private key is $x$, generated exactly as in the El Gamal encryption scheme. In order to encrypt a bit $b$, the sender does the following:
\begin{itemize}
\item If $b=0$ then choose a random $y \gets \mathbb{Z}_q$ and compute $c_1= g^y$ and $c_2=h^y$. The ciphertext is $\langle c_1,c_2\rangle$.
\item If $b=1$ then choose independent random $y,z \gets \mathbb{Z}_q$ and compute $c_1= g^y$ and $c_2=g^z$, and set the ciphertext is $\langle c_1,c_2\rangle$.
\end{itemize}
\end{problem}

\begin{solution}
(a) Show that it is possible to decrypt efficiently given knowledge of $x$.
\vspace{5cm}

\noindent (b) Prove that this encryption scheme is CPA-secure if the decisional Diffie-Hellman problem is hard relative to $\mathcal{G}$.
\vspace{10cm}

\end{solution}

\begin{problem}[10.5]
Suppose that the RSA Assumption fails
``somewhat'' on a particular composite number $N$ and $e$ with
$\gcd(e,\phi(N))=1$, in the sense that there is a $t$-time
algorithm $\mathcal{A}$ such that
\[
\Pr[ \mathcal{A}(x^e) = x \pmod{N} ] > 0.01
\]

Show that there is a $100(\log N)^{100} \cdot t$-time algorithm
$\mathcal{A}'$ that breaks the RSA Assumption completely for $N,e$ in the
sense that
\[
\Pr[ \mathcal{A}'(x^e)= x \pmod{N} ] > 0.99
\]
\textbf{Hint:} Use the fact
that $y^{1/e}r = (yr^e)^{1/e} \pmod{N}$ for every $r \in Z^*_N$.
\end{problem}

\begin{solution}
\vspace{5cm}

\end{solution}

\begin{problem}[10.6]
The natural way of applying hybrid encryption to the El Gamal encryption scheme is as follows. The public key is $pk = \langle \mathbb{G},q,g,h\rangle $ as in the El Gamal scheme, and to encrypt a message $m$ the sender chooses random $k \gets \{0,1\}^n$ and sends
\[ \langle g^r, h^r\cdot k, \mathsf{Enc}_k(m)\rangle, \]
where $r\gets \mathbb{Z}_q$ is chosen at random and $\mathsf{Enc}$ represents a private-key encryption scheme. Suggest an improvement that results in a shorter ciphertext containing only a \emph{single} group element followed by a private-key encryption of $m$.
\end{problem}

\begin{solution}
\vspace{5cm}

\end{solution}

\begin{problem}[10.7]
Let $\widehat{\Pi} = (\widehat{\mathsf{Gen}},f)$ and $\mathsf{hc}$ be as in Construction 1. (See Page 5 in 10tdp-rom.pdf)
\begin{itemize}
\item $\mathsf{Gen}$: as in Construction 1.
\item $\mathsf{Enc}$: on input a public key $I$ and a message $m \in \{0,1\}$, choose a random $x \gets \mathcal{D}_I$ such that $\mathsf{hc}_I(x) = m$, and output the ciphertext $f_I(x)$.
\item $\mathsf{Dec}$: on input a private key $\mathsf{td}$ and a ciphertext $y$ with $y \in \mathcal{D}_{\mathsf{td}}$, compute $x:=f_I^{-1}(y)$ and output the message $\mathsf{hc}_I(x)$.
\end{itemize}
\end{problem}

\begin{solution}
(a) Argue that encryption can be performed in polynomial time, while ensuing that correctness holds with all but negligible probability.
\vspace{5cm}

\noindent (b) Prove that if $\widehat{\Pi}$ is a family of trapdoor permutations and $\mathsf{hc}$ is a hard-core predicate of $\widehat{\Pi}$, then this construction is CPA-secure.
\vspace{10cm}

\end{solution}

\begin{problem}[12.1]
Prove that the existence of a one-time signature scheme for 1-bit messages implies the existence of one-way function.
\end{problem}

\begin{solution}
\vspace{6cm}
\end{solution}

\begin{problem}[12.2]
For each of the following variants of the definition of security for signatures, state whether textbook RSA is secure and prove your answer: 
\end{problem}

\begin{solution}
(a) In this first variant, the experiment is as follows: the adversary is given the public key $pk$ and a random message $m$. The adversary is then allowed to query the signing oracle once on a single message that does not equal $m$. Following this, the adversary outputs a signature $\sigma$ and succeeds if $\mathsf{Vrfy}_{pk}(m,\sigma)=1$. As usual, security is said to hold if the adversary can succeed in this experiment with at most negligible probability.	
\vspace{5cm}

\noindent (b) The second variant is as above, except that the adversary is not allowed to query the signing oracle at all.
\vspace{4cm}

\end{solution}

\begin{problem}[12.3]
Consider the Lamport one-time signature scheme. Describe an adversary who obtains signatures on two messages of its choice and can then forge signatures on any message it likes.
\end{problem}

\begin{solution}
\vspace{4cm}
\end{solution}

\begin{problem}[13.1]
Prove that the pseudorandom function construction (see Page 11 in 10tdp-rom.pdf) is indeed secure in the random oracle model.
\end{problem}

\begin{solution}
\vspace{6cm}

\end{solution}

\end{document}