\documentclass[11pt]{article}

% set 1-inch margins in the document
\usepackage[margin=1in]{geometry}
\usepackage{amsthm}
\usepackage{amsmath}
\usepackage{amssymb,amsfonts}
\theoremstyle{definition}

% include this if you want to import graphics files with /includegraphics
\usepackage{graphicx}
\providecommand{\abs}[1]{\lvert#1\rvert}

% info for header block in upper right hand corner

\newtheorem{problem}{Problem}

\title{HIT --- Cryptography --- Homework 3}

\begin{document}

\maketitle

\begin{problem}
In our attack on a two-round substitution-permutation network, we considered a block length of 64 bits and a network with 16 $S$-boxes that each take a 4-bit input. 
\begin{enumerate}
\item Repeat the analysis for the case of 8 $S$-boxes, each taking an 8-bit input. What is the complexity of the attack now?
\item Repeat the analysis again with a 128-bit block length and 16 $S$-boxes that each take an 8-bit input.
\item Does the block length make any difference?
\end{enumerate}
\end{problem}

\begin{problem}
Show that DES has the property that $DES_k(x) = \overline{DES_{\overline{k}}(\overline{x})}$ for every key $k$ and input $x$ (where $\overline{z}$ denotes the bitwise complement of $z$). This is called the complementarity property of $DES$.
\end{problem}

\begin{problem}
Let $F$ be a pseudorandom function. Show that the following MAC for messages of length $2n$ is insecure: The shared key is a random $k\in \{0,1\}^n$. To authenticate a message $m_1\| m_2$ with $\abs{m_1} =\abs{m_2} = n$, compute the tag $\langle F_k(m_1), F_k(F_k(m_2))\rangle$. 
\end{problem}

\begin{problem}
Let $(\mathsf{Gen},H)$ be a collision-resistant hash function. Is $(\mathsf{Gen},\hat{H})$ defined by $(\hat{H}^s(x) \overset{\text{def}}{=} H^s(H^s(x))$ necessarily collision resistant? Prove your answer. 
\end{problem}

\begin{problem}
For each of following modifications to the Merkle-Damg\r{a}rd transform, determine whether the result is collision resistant or not. If yes, provide a proof; if not, demonstrate an attack.
\begin{enumerate}
\item Modify the construction so that the input length is not included at all (i.e, output $z_B$ and not $z_{B+1} = h^s(z_B\| L)$).
\item Modify the construction so that instead of outputting $z = h^s(z_B\| L)$, the algorithm outputs $z_B\|L$
\item Instead of using an $IV$, just start the computation from $x_1$. That is, define $z_1 := x_1$ and then compute $z_i := h^s(z_{i-1}\|x_i)$ for $i=2,\dotsc,B+1$ and output $z_{B+1}$ as before.
\item Instead of using a fixed $IV$, set $z_0 := L$ and then compute $z_i := h^s(z_{i-1}\|x_i)$ for $i=1,\dotsc,B$ and output $z_B$.
\end{enumerate}
\end{problem}

%\begin{problem}
%Prove that if $f$ is a one-way function, then $g(x_1,x_2) = (f(x_1),x_2)$ where $\abs{x_1} = \abs{x_2}$ is also a one-way function. Observe that $g$ fully reveals half of its input bits, but is nevertheless still one-way.
%\end{problem}

\begin{problem}
Let $f$ be a one-way function. Is $g(x) = f(f(x))$ necessarily a one-way function? What about $g(x) = (f(x),f(f(x)))$? Prove your answers.
\end{problem}

\end{document}